%% (Master) Thesis template
% Template version used: v1.4
%
% Largely adapted from Adrian Nievergelt's template for the ADPS
% (lecture notes) project.


%% We use the memoir class because it offers a many easy to use features.
\documentclass[12pt,a4paper,titlepage]{memoir}

% no book layout now
\setulmarginsandblock{4cm}{4cm}{*}
\setlrmarginsandblock{4cm}{4cm}{*}

%% Packages
%% ========

%% LaTeX Font encoding -- DO NOT CHANGE
\usepackage[OT1]{fontenc}

%% Babel provides support for languages.  'english' uses British
%% English hyphenation and text snippets like "Figure" and
%% "Theorem". Use the option 'ngerman' if your document is in German.
%% Use 'american' for American English.  Note that if you change this,
%% the next LaTeX run may show spurious errors.  Simply run it again.
%% If they persist, remove the .aux file and try again.
\usepackage[ngerman]{babel}

%% Input encoding 'utf8'. In some cases you might need 'utf8x' for
%% extra symbols. Not all editors, especially on Windows, are UTF-8
%% capable, so you may want to use 'latin1' instead.
\usepackage[utf8]{inputenc}

%% This changes default fonts for both text and math mode to use Herman Zapfs
%% excellent Palatino font.  Do not change this.
\usepackage[sc]{mathpazo}

%% The AMS-LaTeX extensions for mathematical typesetting.  Do not
%% remove.
\usepackage{amsmath,amssymb,amsfonts,mathrsfs}

%% NTheorem is a reimplementation of the AMS Theorem package. This
%% will allow us to typeset theorems like examples, proofs and
%% similar.  Do not remove.
%% NOTE: Must be loaded AFTER amsmath, or the \qed placement will
%% break
\usepackage[amsmath,thmmarks]{ntheorem}

%% LaTeX' own graphics handling
\usepackage{graphicx}
\usepackage{lscape}

\graphicspath{ {bilder/} }

%% We unfortunately need this for the Rules chapter.  Remove it
%% afterwards; or at least NEVER use its underlining features.
\usepackage{soul}

%% This allows you to add .pdf files. It is used to add the
%% declaration of originality.
\usepackage{pdfpages}

%% Some more packages that you may want to use.  Have a look at the
%% file, and consult the package docs for each.
\include{extrapackages}

%% Our layout configuration.  DO NOT CHANGE.
\include{layoutsetup}

%% Theorem environments.  You will have to adapt this for a German
%% thesis.
\include{theoremsetup}

%% Helpful macros.
\include{macrosetup}

%% Make document internal hyperlinks wherever possible. (TOC, references)
%% This MUST be loaded after varioref, which is loaded in 'extrapackages'
%% above.  We just load it last to be safe.
\usepackage[linkcolor=black,colorlinks=true,citecolor=black,filecolor=black]{hyperref}


%% longtable
\usepackage{longtable}

%% Document information
%% ====================

\title{Prototyping einer webbasierten Visualisierung zur intuitiven Interaktion mit einem Wissensnetzwerk}
\author{Andreas Waldis, Patrick Siegfried}
\thesistype{}
\advisors{Advisor: Dr.\ Michael Kaufmann}
\department{Departement Informatik}
\date{\today}

\begin{document}

\frontmatter

%% Title page is autogenerated from document information above.  DO
%% NOT CHANGE.
\begin{titlingpage}
  \calccentering{\unitlength}
  \begin{adjustwidth*}{\unitlength-24pt}{-\unitlength-24pt}
    \maketitle
  \end{adjustwidth*}
\end{titlingpage}

%% The abstract of your thesis.  Edit the file as needed.
%\begin{abstract}
  Im Rahmen des Informatikprojekts wurde ein Prototyp für die Visualisierung und Interaktion von \gls{Netzwerk}[en] erarbeitet. Dazu wurden mithilfe der Spezifikation von funktionalen Anforderungen verschiedene \gls{Framework}[s] für die Grundlage der Entwicklung evaluiert. Anschliessend wurde die für den Anwendungsfall beste Wahl getroffen.
  
  Der Prototyp bietet eine Benutzeroberfläche, welche sowohl auf kleinen und grossen Bildschirmen, mit Touch-Gesten oder der Maus und der Tastatur, verwendet werden kann.
  Der resultierende Prototyp wurde als selbständiges Software-Paket entwickelt. Dadurch kann dieser in beliebigen Applikationen verwendet werden. In der letzten Projektphase wurde der Prototyp schlussendlich in den bestehenden \gls{ikc-core} integriert.
\end{abstract}


%% TOC with the proper setup, do not change.
\cleartorecto
\tableofcontents
\mainmatter

%% Your real content!
%\chapter{Einleitung}

Im Hasler-Projekt Intuitive Knowledge Connectivity (IKC) wird ein Prototyp für plattformübergreifendes Wissensnetzwerk erstellt. Die grundlegende Datenbank basiert auf einen Netzwerk (einem gerichteten beschrifteten Property-Graph), hat aber bisher nur eine technische Konsole. Eine ideale grafische Benutzerschnittstelle stellt das Netzwerk oder Teilausschnitte daraus visuell und zweidimensional dar, und stellt beschriftete Knoten und Pfeile grafisch dar. Dies ermöglicht eine einfache und übersichtliche Repräsentation und eine intuitive Interaktion mit den gegebenen Kanten und Knoten. Der bestehende Prototyp implementiert die grundlegenden Datenbankoperationen (C\-R\-U\-D) und die Verknüpfung von Knoten mit Drag and Drop. Auf dieser Kern-Software soll aufgebaut werden, um diese hinsichtlich intuitiver Benutzung zu erweitern, damit mit mehreren Knoten im Netzwerk gleichzeitig und visuell gearbeitet werden kann. Dies soll in erster Linie auf dem Touchscreen (mobil / Tablet) und in zweiter Linie responsive mit dem gleichen Code auch mit Bildschirm, Tastatur und Maus / Touchpad bedienbar sein.
\todo{erweitern, anpassen?}

\chapter{Projektmanagement}

Das folgende Kapitel beschäftigt sich mit der Führung und Kontrolle des Projekts. Das Projektmanagement, wie auch die spätere Entwicklung funktionieren agil. Um möglichst effizient auf Änderungen reagieren zu können, werden entsprechende Hilfsmittel verwendet.
%Im folgenden Kapitel werden Führung und Kontrolle des Projekts aufgezeigt. Das Projektmanagement funktioniert agil, dafür werden geeignete Hilfsmittel eingesetzt. 

\section{Ausgangslage}
Aus dem Forschungsprojekt \gls{Intuitive Knowledge Connectivity}\footnote{\url{https://www.hslu.ch/en/lucerne-university-of-applied-sciences-and-arts/research/projects/detail/?pid=3334}} ging neben einer Publikation\footnote{\citep{ikcpaper:hslu}} auch ein Software-Prototyp\footnote{\url{http://demo.ikc.today/nodeDetail.html}} (\gls{ikc-core}) hervor. Das Projekt beschäftigt sich mit einem intuitiven Umgang mit Daten aus diversen Cloud-Dienstleistern, beispielsweise \gls{Dropbox}, \gls{Evernote} und vielen mehr. Diese werden, ebenfalls \gls{cloudbasiert}, gespeichert und verteilt. Der Mehrwert entsteht durch die Verknüpfung der verschiedenen Daten zu einem gesamtheitlichen Ganzen. Diese Verknüpfungen sind be\-nutzer\-bas\-iert und werden mithilfe einer Graph-Datenbank gehandhabt.

Da beide Projektmitarbeiter ebenfalls am Forschungsprojekt tätig sind, ist das Grundlagewissen bereits grösstenteils vorhanden. Die Arbeit am Forschungsprojekt und am \gls{PAWI} verlaufen parallel weiter. Um eine adäquate Trennung der beiden Projekte zu gewährleisten, wird im \autoref{sec:scope} näher auf die Gegebenheiten eingegangen.

\subsection{Internationale Projektorganisation}
\label{subsec:international}
Damit Andreas Waldis ein Team-Mitglied während des Projekts in den Vereinigten Staaten im Austauschsemester weilt, müssen einige zusätzliche organisatorische Massnahmen ergriffen werden. Diese umfassen die folgenden Tools und Arbeitsweisen:
\begin{itemize}
\item Die Zeitverschiebung zwischen Luzern (\textit{UTC+01:00}) und West Lafayette (\textit{UTC-05:00}) beträgt sechs Stunden. Daher werden Sitzungen überwiegend nachmittags nach 14:00 angesetzt. 
\item Projektplanung: Die Projektdauer wurde für diesen speziellen Fall um sechs Wochen auf Ende Januar erweitert. Dies ermöglicht die verschiedenen Teammitglieder entsprechend ihrer Auslastung zu arbeiten. Insbesondere für Andreas Waldis, welcher bereits im Dezember die Abschlussprüfungen hat. 
\end{itemize}

\subsection{Werkzeuge}

Es werden verschiedene Werkzeuge verwendet, um eine möglichst optimale Kommunikation und Kollaboration zu gewährleisten. Dazu gehören:
\begin{enumerate}
% \item \textbf{Slack}: Ein Messenger, welcher für die Verwendung im professionellen Umfeld ausgelegt ist. So können beispielsweise verschiedenet Cloud-Services integriert werden.\footnote{\url{https://slack.com/}}
\item \textbf{Skype}\footnote{\url{https://www.skype.com/de/}} \& \textbf{Discord}\footnote{\url{https://discordapp.com/}}: Beides Werkzeuge für Video- und Audio-Sitzungen. Skype ist wohlbekannt. Discord hingegegen ist spezialisiert für Gamer und beinhaltet ein umfangreicheres Chat-System.
\item \textbf{Targetprocess}\footnote{\url{https://www.targetprocess.com}}: Vergleichbar mit dem bekannten ScrumDo\footnote{\url{http://www.scrumdo.com}}. Im Gegensatz dazu verfügt Targetprocess jedoch über eine moderne Benutzeroberfläche und scheint für den vorliegenden Verwendungszweck allgemein geeigneter zu sein.
\item \textbf{Sharelatex}\footnote{\url{https://www.sharelatex.com}}: Online Kollaborationstool für Latex. Anstelle von lokalen Installationen in einem gemeinsamen Versions\-verwalt\-ungs\-sys\-tem, wird einfach direkt online am selben Dokument gearbeitet.
\item \textbf{draw.io}\footnote{\url{https://www.draw.io}}: Werkzeug, um online Diagramme zu zeichnen. Diese können ebenfalls gemeinsam und gleichzeitig gezeichnet werden. Zusätzlich ist es ideal, um schnell und einfach alles Mög\-liche zu zeichnen.
\end{enumerate}

\section{Scope}
\label{sec:scope}

Die Schwierigkeit liegt zu einem grossen Teil in der Abgrenzung zum parallellaufenden Projekt \gls{IKC}. Grundsätzlich gehört jegliche Arbeit im zweidimensionalen Bereich zum \gls{PAWI}. Der bestehende \gls{ikc-core} wird soweit vorbereitet, dass die Visualisierung ohne Zusatzaufwand integriert werden kann. Hierzu gehören Arbeiten betreffend Schnittstellen, Absenden von allfälligen \gls{Event}s und dergleichen. Das Ziel ist die Entwicklung eines Zusatzmoduls zum \gls{ikc-core}, welches mittels definierten Schnittstellen ein- und angebunden werden kann. Die genaue Spezifikation ist im Lösungskonzept noch zu erarbeiten.

Weiter ist zu betonen, dass die intuitive Bedienung (\gls{Usability} oder auch \gls{User Experience}) nach bestem Wissen und Gewissen angestrebt wird. Allerdings handelt es sich bei den Projektmitarbeitern um Informatiker ohne grundlegende Kenntnisse in den angesprochenen Bereichen. Die Benutzeroberfläche wird darum nach dem subjektiven Befinden des Projektpartners und den beiden Projektmitarbeitern entwickelt.

\section{Projektstrukturplan}
Die \autoref{fig:projektstrukturplan} gewährt einen Überblick über das Projekt. Sie stellt die wichtigsten Bereiche und Phasen dar, in welche die Arbeit grob eingegliedert werden kann:
\begin{enumerate}
    \item Die \textbf{Projektführung} beinhaltet die Planung des Vorhabens über den gegebenen Zeitraum. Ständige Kontrolle des Ist- gegenüber dem Soll-Zustand kann gegebenenfalls zur Steuerung oder Anpassungen des Zeitplans führen. Im Gegensatz zu den anderen Bereichen wird die Projektführung über die volle Projektdauer ausgeführt. Da das Projekt agil organisiert ist, liegt das Augenmerk auf der Priorisierung der Anforderungen.
    \item In der \textbf{Konzeption} werden neben den Anforderungen auch mög\-li\-che Lösungsansätze in den Bereichen Architektur, Schnittstellen und \gls{User Interface}/\gls{User Experience} gesammelt.
    \item Nach der Recherche von Lö\-sungs\-an\-sätz\-en muss allenfalls die Auswahl an Mög\-li\-chkei\-ten eingeschränkt werden. Eine anschliessende Testphase erleichtert die \textbf{Evaluation und Auswahl} der Lösungsvariante.
    \item Nun beginnt die Phase der eigentlichen \textbf{Entwicklung}. Auf Basis der bestehenden Tests wird die Visualisierung mit allen erforderlichen Komponenten erarbeitet. Der Schwerpunkt liegt auf der Gestik und dem \gls{Responsive Design}. Ein abschliessendes Testing des Systems stellt sicher, dass die Integration beginnen kann.
    \item Sobald die Visualisierung reibungslos funktioniert, wird sie in den bestehenden \gls{ikc-core} \textbf{integriert}. Alle Anforderungen werden erfüllt und anschliessend getestet.
    \item Nachdem die Entwicklungsarbeiten abgeschlossen sind, folgt der \textbf{Projektabschluss}. Dabei wird die endgültige Version des Projektreports erstellt und die Abschlusspräsentation gehalten.
\end{enumerate}

\newpage

\begin{landscape}
\begin{figure}[ht]
\centering
\includegraphics[width=1.5\textwidth]{Projektstrukturplan}
\caption{Projektstrukturplan}
\label{fig:projektstrukturplan}
\end{figure}
\end{landscape}

\newpage

\section{Rahmenplan}
Die Rahmenplanung, basierend auf dem Projektstrukturplan (\autoref{fig:rahmenplan}), repräsentiert die zeitliche Planung des Projekts. Dabei werden Kalenderwochen anstelle von Daten oder Schulwochen verwendet. Dies aufgrund des internationalen Rahmens, der damit verbundenen Zeitverschiebung und unterschiedlichen Stundenplänen. Enthalten sind alle Projektphasen, Sprints und Meilensteine, als auch alle Lieferobjekte welche im \autoref{lieferobjekte} weiter ausgeführt werden. Die Dauer der Sprints wird bewusst unterschiedlich ausgestaltet, um den verschiedenen Projektphasen und deren Inhalten Rechnung zu tragen.
%Die Sprints dauern absichtlich unterschiedlich lange, das deshalb, weil die Länge auf\-grund der verschiedenen Projektphasen und deren Inhalt zugeordnet worden ist. Weiter werden administrative Elemente durch blaue Färbung und Entwicklungs-Elemente durch rote Färbung gekennzeichnet.

Eine grosse Rolle in der Rahmenplanung spielen die Meilensteine. Sie unterteilen das Projekt in Phasen, welche dadurch klar voneinander getrennt sind. Ebenfalls sind sie eine wichtige Orientierungshilfe im Projekt und weisen den Weg damit das Projekt erfolgreich abgeschlossen werden kann. Die \autoref{tab:meilensteine} listet die Meilensteine auf.


\begin{longtable}{|p{1cm}|p{2cm}|p{8.5cm}|}
  \hline
    ID & Datum &  Beschreibung \\\hline
    M1 & 19.09.2016 & Administrativer Meilenstein: Kickoff\\\hline
    M2 & 09.10.2016 & Administrativer Meilenstein: Projektplanung abgeschlossen\\\hline
    M3 & 16.10.2016 & Entwicklung Meilenstein: Schnittstellen definiert\\\hline
    M4 & 06.11.2016 & Entwicklung Meilenstein: Evaluation Entscheid\\\hline
    M5 & 25.12.2016 & Entwicklung Meilenstein: Funktionsfähige Oberfläche umgesetzt\\\hline
    M6 & 15.01.2017 & Entwicklung Meilenstein: Integration in \gls{ikc-core} abgeschlossen\\\hline
    M7 & 22.01.2017 & Administrativer Meilenstein: 95\% erreicht\\\hline
    M8 & 30.01.2017 & Administrativer Meilenstein: PAWI Bericht Abgabe\\\hline
    M9 & 31.01.2017 & Administrativer Meilenstein: Präsentation\\\hline
    \caption{Meilensteine}
  \label{tab:meilensteine}
\end{longtable}

\newpage

\begin{landscape}
\begin{figure}[ht]
\centering
\includegraphics[width=1.7\textwidth]{Rahmenplan}
\caption{Rahmenplan}
\label{fig:rahmenplan}
\end{figure}
\end{landscape}

\newpage

\section{Projektziele} \label{projektziele}
Projektziele werden definiert, um den Erfolg an einigen ausgewählten Punkten zu überprüfen und sicherstellen zu können. Sie wurden in Absprache mit dem Kunden definiert. Die Ziele sind in der folgenden \autoref{tab:projekt-ziele} aufgelistet.

\begin{longtable}{|p{1cm}  | p{10.5cm}|}
  \hline
    ID &  Beschreibung \\\hline
    Z1 & Ergänzung zur bestehenden Benutzeroberfläche.\\\hline
    Z2 & Entwicklung nach dem Konzept \gls{mobile first}.\\\hline
    Z3 & Intuitive, visuelle und effiziente Interaktion mit einem \gls{Netzwerk}.\\\hline
    Z4 & Übersichtliche Visualisierung der \gls{Node}[s] und \gls{Link}[s].\\\hline
    Z5 & Erweiterung der Funktionalität mittels Sichten (\gls{View}s).\\\hline
    \caption{Projektziele}
  \label{tab:projekt-ziele}
\end{longtable}

\section{Anforderungen} \label{anforderungen}

Für die weitere Unterteilung in Arbeitspakete und \textit{Stories} werden die Anforderungen zunächst in Prosa gesammelt. Diese entstammen dem Kundenworkshop und der Aufgabenstellung, sind im Sinn der Projektziele (\autoref{projektziele}). Die Anforderungen werden unterschieden in funktionale und nicht-funktionale Anforderungen. Die funktionalen Anforderungen definieren direkt die Eigenschaften, (\autoref{tab:funktionale-anforderungen}). Im Gegensatz dazu definieren nicht-funktionale Anforderungen die Leistung und die Randbedingungen, aufgelistet in \autoref{tab:nicht-funktionale-anforderungen}.

Die Priorisierung erfolgt nach dem \gls{MoSCoW-System}:

\begin{longtable}{|p{1.5cm} | p{2.5cm} | p{7.2cm}|}
  \hline
    \# & Priorität & Beschreibung \\\hline
    M & Must Have & Bei dieser Anforderung handelt es sich um ein Muss, höchste Priorität.\\\hline
    S & Should Have & Diese Anforderung wird erwartet, normale Priorität.\\\hline
    C & Could Have & Tiefste Priorität, desiderata.\\\hline
    \caption{MosCow-Priorisierung}
  \label{tab:moscow}
\end{longtable}


\begin{longtable}{|p{1.5cm} | p{1.5cm} | p{8.1cm}|}
  \hline
    ID & Priorität & Beschreibung \\\hline
    A1.1 & M & Die zugrundeliegende Datenbasis kann mittels eines \gls{Netzwerk}[s] visualisiert werden. Jenes besteht aus Knoten und Kanten, welche gerichtet und beschriftet sind.\\\hline
    A1.2 & M & Die zu erarbeitende Visualisierung kann bidirektional mit dem \gls{ikc-core} interagieren. Änderungen am \gls{ikc-core} sind in der Visualisierung sichtbar. Es ist jedoch auch möglich Änderungen direkt in der Visualisierung vorzunehmen.\\\hline
    A1.3 & S & Mittels der Visualisierung können die grundlegenden Datenbankoperationen \textit{CREATE}, \textit{READ}, \textit{UPDATE} und \textit{DELETE} (CRUD) teilweise direkt auf der Datenbasis des \gls{ikc-core}[s] angewendet werden.\\\hline
    A1.3.1 & S & Es können sowohl Knoten aus der Visualisierung als auch aus der Datenbasis\footnote{Hier ist die Datenbasis des \gls{ikc-core}[s] gemeint.} gelöscht werden. Diese Vorgänge können klar unterschieden werden.\\\hline
    A1.3.2 & M & Knoten ohne Kanten können in der Visualisierung abgebildet werden.\\\hline
    A1.3.3 & M & Knoten können in der Visualisierung erstellt werden.\\\hline
    A1.4 & M & Der Prototyp kann auf Smartphones und Tablets benutzt werden.\\\hline
    A1.5 & S & Der Prototyp kann auf Laptops und Desktop-Computern benutzt werden.\\\hline    
    A1.6 & M & Mit Hilfe verschiedener Kriterien (beispielsweise Kontext oder Nachbarschaft) kann ein Teil eines \gls{Netzwerk}[s] visualisiert werden.\footnote{Die Visualisierung eines Teils eines \gls{Netzwerk}[s]  wird später als \gls{View} bezeichnet.}\\\hline 
    A1.6.1 & S & Diese Visualisierungen (ein Teil des \gls{Netzwerk}[s]) können unabhängig von der Datenbasis persistiert werden. Es können, zusätzlich zu den Knoten und Kanten, auch deren relative Positionen in der Visualisierung gespeichert werden.\\\hline 
    A1.7 & C & Der Prototyp kann (unter anderem) mit \gls{Drag'n'Drop} bedient werden.\\\hline
    A1.7.1 & S & Bidirektionale und unbeschriftete Verknüpfungen zwischen visualisierten Knoten können erstellt werden.\\\hline
    A1.7.2 & M & Ein, in der bestehenden Benutzeroberfläche\footnote{\gls{ikc-core}}, mittels der Suche gefundener Knoten kann direkt in die Visualisierung übernommen werden.\\\hline 
    A1.7.3 & S & Ein mittels der Suchfunktion gefundener Knoten kann auf einen bestehenden Knoten gezogen werden, um mit diesem eine Kante zu bilden.\\\hline 
    A1.7.4 & S & Knoten können innerhalb des Diagramms frei positioniert werden.\\\hline 
    A1.8 & M & Die Visualisierung kann innerhalb der bestehenden \gls{ikc-core} Oberfläche genutzt werden.\\\hline 
    A1.9 & S & In der Visualisierung könnnen auch Knoten mit mehr als sieben Kindknoten dargestellt werden: Sind mehr als sieben Kindknoten vorhanden, wird anstelle einer Repräsentation jedes einzelnen Knotens auf eine Liste gewechselt. Diese Liste kann zusätzlich mit einer Such- und oder Filterfunktion versehen werden.\\\hline  
    A1.10 & C & Die von einem Knoten aus- und eingehenden Kanten können auf- und zugeklappt (sichtbar-unsichtbar) werden.\\\hline
    A1.10.1 & C & Einzelne Kanten können individuell zugeklappt/ausgeblendet werden.\\\hline
     
    \caption{Funktionale Anforderungen}
  \label{tab:funktionale-anforderungen}
\end{longtable}

\begin{longtable}{|p{1.5cm} | p{1.5cm} | p{8.1cm}|}
  \hline
    ID & Priorität & Beschreibung \\\hline
    A2.1 & M & Bestehende Komponenten des \gls{ikc-core}[s] werden, wo möglich, wiederverwendet bzw. erweitert (nachhaltige Entwicklung).\\\hline
    A2.2 & M & Die Visualisierung kann in verschiedenen Projekten wiederverwendet werden.\\\hline
    A2.3 & M & Die Visualisierung wird im \gls{ikc-core} integriert (Deployment).\\\hline
    A2.4 & M & Die Visualisierung wird parallel zum \gls{ikc-core} weiterentwickelt. Die Basis für die Visualisierung bildet ein Klon der bestehenden Codebasis aus dem Versionkontrollsystem. Ist die Entwicklung abgeschlossen, werden die beiden Entwicklungsstände mittels eines neuen Astes (Branch) zusammengeführt.\\\hline
    A2.5 & M & Jegliche Daten werden auf der Dropbox des Benutzers persistiert.\\\hline
    A2.6 & S & Es wird eine intuitive, effiziente Bedienung angestrebt. Ein Mass für die Effizienz: $\frac{\text{Taps}}{\text{Task}}$\\\hline
    A2.7 & M & Randbedingungen 180h pro Person\\\hline
    A2.8 & S & Die Arbeit können in einem Arbeitsjournal, mindestens mit Angaben von Datum, Anzahl Stunden, Arbeitsschritt/Thema, nachverfolgt werden.\\\hline
    \caption{Nicht funktionale Anforderungen}
  \label{tab:nicht-funktionale-anforderungen}
\end{longtable}

\section{Risikoanalyse}\label{risikoanalyse}

In folgender \autoref{tab:risikoanalyse} werden mögliche Risiken behandelt. Die Wahrscheinlichkeit ist mit P abgekürzt. R steht für Risiko und S für den Schaden, welcher mittels $P*R=S$ berechnet wird.

\clearpage

\begin{longtable}{|p{0.5cm} | p{7cm} | p{1cm}|  p{1cm}|  p{1cm}|}
  \hline
    ID & Beschreibung &  P & R & S \\\hline
    R1 & Wie in \autoref{subsec:international} bereits angesprochen wird das Projekt in einem \textbf{internationalen Rahmen} durchgeführt. Dies birgt einige Herausforderungen und bringt auch Riksiken mit sich: Die grösste Schwierigkeit bildet sicherlich die Zeitverschiebung. Aber auch die geographische Distanz und die lediglich im digitalen Rahmen gehaltenen Sitzungen sind nicht zu unterschätzen.\newline
    Um dieser Problematik zu entgegnen, werden diverse Kommunikationswege, vorwiegend verschiedenen On\-line-Platt\-for\-men, benutzt. Diese sollen trotz den Gegebenheiten die Kollaboration und den Austausch unterstützen und fördern. & 1 & 3 & 3\\\hline
    R2 & Die Abgrenzung vom laufenden Projekt \textbf{\acrshort{IKC}} (vgl. \autoref{sec:scope}) ist klar festzulegen und einzuhalten. So können Überschneidungen und Unklarheiten verhindert werden.\newline
    Vollständige und detaillierte Arbeitsjournale, wie auch Protokolle bieten dabei eine wichtige Hilfestellung.  & 2 & 1 & 2\\\hline
    R3 & Die Zahl der Anforderungen ist hoch. Im Rahmen der \textbf{360 zu leistenden Stunden} ist es wichtig, sich nicht in Details oder in der Ausarbeitung jeglicher Finessen zu verlieren.\newline
    Darum ist es umso wichtiger, die Anforderungen klar zu priorisieren und auch entsprechend abzuarbeiten.  & 3 & 1 & 3\\\hline
    \caption{Risikoanalyse}
  \label{tab:risikoanalyse}
\end{longtable}

\section{Lieferobjekte}\label{lieferobjekte}

Neben den in der Aufgabenstellung vorgegebenen Lieferobjekte (\autoref{tab:set-lieferobjekte}) sind noch zusätzliche, interne Lieferobjekte (\autoref{tab:add-lieferobjekte}) festlegt. Diese sind lediglich als Unterstützung der Projektkontrolle, eine Art Orientierungshilfe, gedacht.

\begin{longtable}{|p{1cm} | p{2cm} | p{8.1cm}|}
  \hline
    ID & Datum &  Beschreibung \\\hline
    L1 & 30.09.2016 & Projektplanung.\\\hline
    L2 & 07.10.2016 & Anforderungskatalog.\\\hline
    L3 & 17.10.2016 & Lösungskonzept inkl. Schnittstellendefinition.\\\hline
    L4 & 30.01.2017 & Funktionsfähige Software mit folgenden Eigenschaften:
    \begin{itemize}
        \item Integriert bestehender Prototyp auf Basis \hyperref[react]{\textit{React}} mit allen \textbf{\textit{CRUD}}\footnote{Create Read Update Delete} Funktionen und Schnittstellen zu \gls{Dropbox} und \gls{Evernote}.
        \item Möglichkeit zur visuellen Interaktion mit einem 
            Teil eines \gls{Netzwerk}[s]: Knoten können per \gls{Drag'n'Drop} verknüpft werden
        \item Die Applikation läuft auf dem Mobile, Tablet, Laptop und Desktop (in Priorität der Reihenfolge) $\rightarrow$ responsive CSS Template\footnote{\gls{Responsive Design}} verwenden.
    \end{itemize}
    \\\hline
    L5 & 30.01.2017 & Dokumentierter Sourcecode (für Methoden und Parameter).\\\hline
    L6 & 30.01.2017 & Benutzerhandbuch.\\\hline
    L7 & 30.01.2017 & \gls{PAWI}-Bericht.\\\hline
    \caption{Vorgegebene Lieferobjekte}
  \label{tab:set-lieferobjekte}
\end{longtable}
 
\begin{longtable}{|p{1cm} | p{2cm} | p{8.1cm}|}
  \hline
    ID & Datum &  Beschreibung \\\hline
    L1' & 07.11.2016 & Abgeschlossene Evaluation der verschiedenen \gls{Framework}[s] zur Umsetzung der Visualisierung.\\\hline
    L2' & 26.12.2016 & Funktionsfähige Visualisierung gemäss den funktionalen Anforderungen A1.3, A1.4, A1.5, A1.6, A1.7, A1.9 und A1.10. Bereit für die Integration in den \gls{ikc-core}.\\\hline
    L3' & 13.01.2017 & Abgeschlossene Integration der Visualisierung in den \gls{ikc-core} und Erfüllung aller Anforderungen insbesondere A1.1, A1.2.\\\hline
    \caption{Zusätzliche, interne Lieferobjekte}
  \label{tab:add-lieferobjekte}
\end{longtable}

\section{Stories}
Die User-Stories repräsentieren alle Arbeitspakete, welche über die gesamte Projektdauer geplant wurden. Diese werden nicht nur im klassischen Sinne für die Klassifizierung von Applikationsfunktionen, sondern auch für konzeptionelle Aufgaben verwendet. Insgesamt haben wurde der Aufwand mit \textbf{256} Punkte beziffert, wobei ein Punkt circa einer Stunde entspricht. Dies entspricht auch etwa dem resultierenden Aufwand von \textbf{276} Punkten. Der Mehraufwand konnte dank einiger Reserven gut kompensieren werden. Dieser entstand vor allem in der Umsetzung der \gls{Drag'n'Drop} Gesten und dem Datenaustausch zwischen \gls{ikc-core} und Visualisierung. Alle User-Stories sind in der \autoref{user-stories} detailliert aufgelistet und die weiterführenden Beschreibungen sind in der \autoref{user-stories-desc} zu finden.
\begin{longtable}{|p{0.6cm}|P{3.5cm}|p{1.4cm}|p{1.4cm}|p{2.4cm}|p{1.1cm}|}
\hline
ID  & Name & Geplant & Effektiv & Phase & Sprint\\ \hline
S1 & Defintion Schnittstellen           & 10 pt             & 12 pt               & Evaluation  & 2.1 \\ \hline
S2 & Grobauswahl                        & 20 pt             & 17 pt               & Evaluation  & 2.1 \\ \hline
S3 & Detail Beurteilung zwei \gls{Framework}[s] & 4 pt               & 5 pt                & Evaluation  & 2.3 \\ \hline
S4 & Detail Beurteilung zwei \gls{Framework}[s] & 4 pt               & 4 pt                & Evaluation  & 2.2 \\ \hline
S5 & Definition Standard Szenario       & 4 pt               & 3 pt                & Evaluation  & 2.1 \\ \hline
S6 & Mook Up erstellen                  & 8 pt               & 10 pt               & Evaluation  & 2.3 \\ \hline
S7 & Basic Setup                        & 3 pt               & 4 pt                & Entwicklung & 3.1 \\ \hline
S8 & \gls{Drag'n'Drop}                        & 20 pt              & 29 pt               & Entwicklung & 3.4 \\ \hline
S9 & PositionUpdate                     & 4 pt               & 4 pt                & Entwicklung & 3.1 \\ \hline
S10 & NewLink                            & 4 pt               & 4 pt                & Entwicklung & 3.2 \\ \hline
S11 & Core Context Menu                    & 15 pt              & 22 pt               & Entwicklung & 3.1 \\ \hline
S12 & Node Context Menu                    & 25 pt              & 23 pt               & Entwicklung & 3.2 \\ \hline
S13 & LinkCollapse                       & 10 pt              & 11 pt                & Entwicklung & 3.3 \\ \hline
S14 & Show/Hide Labels                   & 5 pt               & 5 pt                & Entwicklung & 3.4 \\ \hline
S15 & Integration IKC                    & 45 pt              & 38 pt               & Integration & 4.1 \\ \hline
S16 & Datenaustausch                     & 20 pt              & 26 pt               & Integration & 4.2 \\ \hline
S17 & Persistenz                         & 20 pt              & 23 pt               & Integration & 4.2 \\ \hline
S18 & Dialoge                            & 20 pt              & 22 pt               & Integration & 4.1 \\ \hline
S19 & SearchFields                       & 15 pt              & 14 pt               & Integration & 4.1 \\ \hline\hline
 & \textbf{Total}                       & \textbf{256 pt}             & \textbf{276 pt}               &  &  \\ \hline
    \caption{User Stories}
 \label{user-stories}
\end{longtable}

\section{Testkonzept}
Basierend auf den \hyperref[anforderungen]{Anforderungen} wurden die verschiedenen Testfälle definiert. Diese sind hier zusammengefasst. Konkret handelt es sich um die folgenden Testfälle (\autoref{tab:testkonzept}), welche im \autoref{tests} genauer beschrieben sind.

\begin{longtable}{|p{1cm} | P{6cm} |}
  \hline
    ID & Kurzbeschrieb \\\hline
    T1 & Neuer \gls{Node} erstellen und darstellen.\\\hline
    T2 & Neuer \gls{Node} erstellen, darstellen und mit einem bereits dargestellten \gls{Node} verbinden.\\\hline
    T3 & Bestehender \gls{Node} in der Visualisierung darstellen.\\\hline
    T4 & Bestehender \gls{Node} mit einem bereits dargestellten \gls{Node} verbinden.\\\hline
    T5 & Bestehender \gls{Node} in der Visualisierung darstellen (\gls{Drag'n'Drop}).\\\hline
    T6 & Bestehender \gls{Node} mit einem bereits dargestellten \gls{Node} verbinden (\gls{Drag'n'Drop}).\\\hline
    T7 & \gls{Node} ausblenden.\\\hline
    T8 & Alle ausgehenden \gls{Link}[s] ausblenden.\\\hline
    T9 & Alle ausgehenden \gls{Link}[s] einblenden.\\\hline
    T10 & Einzelner ausgehender \gls{Link} einblenden.\\\hline
    T11 & Verschiedene ausgewählte \gls{Link}[s] ausblenden.\\\hline
    T12 & Neue \gls{View} erstellen.\\\hline
    T13 & Existierende \gls{View} öffnen.\\\hline
    T14 & \gls{Node} Informationen in der Datenbasis aktualisieren.\\\hline
    T15 & \gls{Node} in der Datenbasis löschen.\\\hline
    T16 & \gls{Link} in der Datenbasis löschen.\\\hline
    T16 & \gls{Link} in der Datenbasis erstellen.\\\hline
    T17 & \gls{Node} mit mehr als sieben ausgehende \gls{Link}[s] darstellen.\\\hline
    T18 & Label der \gls{Link}[s] aus- und einblenden. \\\hline
    \caption{Testfälle}
  \label{tab:testkonzept}
\end{longtable}




%\chapter{Lösungsdesign}

Das Lösungsdesign beinhaltet die Grundlagen für die erfolgreiche Umsetzung des Prototypen. Dieses Kapitel beinhaltet die definierten Schnittstellen sowohl die Auswahl des Frameworks für die Darstellung. Ein besonderes Augenmerk wird auf die Interaktion zwischen der Visualisierung und dem \textit{ikc-core} gelegt, besonders auf die reduktion der Kopplung zwischen den beiden Komponenten. 

\section{Technische Ausgangslage}
Der \textit{ikc-core} funktioniert momentan mit einer puristischen Benutzeroberfläche. Die Applikationslogik und der Umgang mit der Datenbasis existieren somit bereits. Wie auf dem Komponentendiagramm (\autoref{fig:komponentendiagramm}) ersichtlich, nutzt der \textit{ikc-core} die Visualisierung (graph-visualization). Jedoch müssen dazu verschieden Schnittstellen auf der Visualisierung implementiert werden. Diese werden dann von der Visualisierung genutzt um Operation an den \textit{ikc-core} zu delegieren. Dies kann als Nutzungsvertrag zwischen der Visualisierung und dem Komponente, welche sie verwendet verstanden werden. Als Beispiel küm\-mert es die Visualisierung wenig, wie und wo ein Node gespeichert wird. Es wird nur die jeweilige Methode ausgeführt und alles andere geschieht ausserhalb der Vi\-su\-ali\-si\-erung. Mehr Details sind im \autoref{sec:architektur} zu finden.

\begin{figure}[htbp]
\centering
\includegraphics[width=0.6\textwidth]{components}
\caption{Komponentendiagramm}
\label{fig:komponentendiagramm}
\end{figure}

\section{Architektur}
\label{sec:architektur}

Das Klassendiagramm (\autoref{fig:klassendiagramm}) gewährt einen detaillierteren Überblick über die Architektur der Visualisierung. Es zeigt die Schnittstellen zum \textit{ikc-core} und die wichtigsten Komponenten auf.

\todo[inline]{Mehr Details?}

\begin{landscape}
\todo[inline]{Diagramm neuzeichnen (vektorisiert) und updaten (Signaturen)}

\begin{figure}[htbp]
\centering
\includegraphics[width=1.6\textwidth]{architecture-overview}
\caption{Klassendiagramm}
\label{fig:klassendiagramm}
\end{figure}
\end{landscape}


\section{Schnittstellen} \label{schnittstellen}

Die hier gezeigten Schnittstellen dienen dem Austausch und der Interaktion mit dem \textit{ikc-core}. Um die Visualisierung in einer bestehenden Umgebung verwenden zu können, müssen die erforderlichen Teile implementiert werden.

Die \autoref{fig:integration-ikc-core} zeigt einen Teil einer möglichen Integration, beispielsweise in den bestehenden \textit{ikc-core}. Die rechte Seite stellt einen Ausschnitt des \textit{ikc-visual}-Paketes an. Die Klasse \textit{GraphScreen} (\autoref{GraphScreen}) ist zu\-stä\-ndig für die Visualisierung des Netzwerkes. Um den vollen Funktionsumfang zu bieten, benutzt sie, unter anderem, die Schnittstelle \textit{DialogFactory} (!!!REFERENZ!!!). Diese soll den Umgang mit Dialogfenstern ermöglichen. Da dies zu den grundlegenden Funktionen der Visualisierung zählt, befindet sich diese Schnittstelle direkt im \textit{ikc-visual}-Paket. Bei der Integration der Visualisierung in eine bestehende Umgebung gilt es somit, die verschiedenen Schnittstellen entsprechend zu implementieren.

Die Klasse \textit{GraphVisualisation} verwendet die Klasse \textit{GraphScreen} aus der Visualisierung. Folglich muss beispielsweise die Schnittstelle \textit{DialogFactory} implementiert werden. Dies wird mit der Klasse \textit{GraphDialogFactory} realisiert.

Ähnlich wie beim oben beschriebenen Beispiel gibt es weitere Voraussetzungen des \textit{GraphScreen}, welche bei einer Integration beachtet werden sollten. Diese werden nachfolgenden genau erläutert.

\todo[inline]{Referenzen zu folgenden Unterkapiteln}

Es folgt eine Beschreibung der im \textit{ikc-visual}-Paket enthaltenen Klassen und Schnittstellen. 
\begin{figure}[htbp]
\centering
\includegraphics[width=0.7\textwidth]{architecture}
\caption{Integration ikc-core}
\label{fig:integration-ikc-core}
\end{figure}

\subsection{NodeComponentFactory}
Mit dem \textit{Factory-Pattern} wird die Objekterzeugung von der tatsächlichen Implementation entkoppelt. Dies wird verwendet um die Komponentenrepräsentationen der Knoten und Kanten auszuliefern (\autoref{lst:nodecomponentfactory}).

\begin{listing}[htbp]
\inputminted[
frame=lines,
framesep=2mm,
baselinestretch=1.2,
linenos,
breaklines=true
]{js}{sourcecode/common/interfaces/NodeComponentFactory.ts}
\caption{NodeComponentFactory-Interface}
\label{lst:nodecomponentfactory}
\end{listing}

\subsection{NodeInformationProvider}
Die Visualisierung benötigt zum Darstellen des Netzwerk diverse Informationen zu den einzelnen Knoten und Kanten. Diese werden über die beiden Methoden \texttt{getNodeTitle} und \texttt{getNodeEdgesIds} geliefert. Die Identifikation erfolgt jeweils über eine eindeutige Identifikationsnummmer (\autoref{listing:nodeinformationprovider}).

Teile der Visualisierung sollen mittels \textit{Drag and Drop} bedient werden können. Folgend ein Beispiel, wie die Schnittstelle in diesem Anwendungsfall benutzt werden kann:

Beim \textit{Drag and Drop} wird üblicherweise ein \textit{Event} beim Loslassen des gepackten Elements ausgelöst. Das Element soll eine eindeutige Identifikationsnummer (ID) enthalten. Dies ist aus dem bestehenden \textit{IKC-Core} vorausgesetzt. Wird das Element nun innerhalb der Visualisierung losgelassen, enthält der ausgelöste \textit{Event} so auch diese ID. Diese kann zusammen mit dem \textit{Event} abgefangen und wiederverwendet werden. Beispielsweise können nun Informationen über den \textit{NodeInformationProvider} bezogen werden.

\begin{listing}[htbp]
\inputminted[
frame=lines,
framesep=2mm,
baselinestretch=1.2,
linenos,
breaklines=true
]{js}{sourcecode/common/interfaces/NodeInformationProvider.ts}
\caption{NodeInformationProvider-Interface}
\label{listing:nodeinformationprovider}
\end{listing}

\subsection{OperationService}
Die \textit{OperationService}-Schnittstelle ermöglicht der Visualisierung mit der Datenbasis zu interagieren. Wird beispielsweise in der Visualisierung ein neuer Knoten erstellt, muss dieser auch auf der Datenbasis erzeugt werden (\autoref{listing:operationservice}). 

Lesezugriffe erfolgen aber über den zuvor genannten \textit{NodeInformationProvider} (\autoref{listing:nodeinformationprovider}).

\begin{listing}[htbp]
\inputminted[
frame=lines,
framesep=2mm,
baselinestretch=1.2,
linenos,
breaklines=true
]{js}{sourcecode/common/interfaces/OperationService.ts}
\caption{OperationService-Interface}
\label{listing:operationservice}
\end{listing}

\subsection{View}
Die \textit{View} ist eine Abstraktion der Visualisierung. Sie hält einerseits Informationen zu der einzelnen Sicht, welche Kanten und Knoten sind sichtbar. Andererseits ermöglichst sie auch das persistieren der Sicht. Dies erfolgt über eine \textit{JSON}-Repräsentation (\autoref{listing:view}).

\begin{listing}[htbp]
\inputminted[
frame=lines,
framesep=2mm,
baselinestretch=1.2,
linenos,
breaklines=true
]{js}{sourcecode/common/interfaces/View.ts}
\caption{View-Interface}
\label{listing:view}
\end{listing}

\section{Komponenten}\label{komponenten}

Die \textit{React}-Komponenten werden seitens der Visualisierung implementiert. Der \textit{Graphscreen} (\autoref{listing:graphscreen}) beinhaltet hier \textit{DiagramNodes} (\autoref{listing:diagramnode}) und \textit{DiagramEdges} (\autoref{listing:diagramedge}). Für die grafische Repräsentation der verschiedenen Komponenten werden direkt \textit{React}-Komponenten verwendet (vgl. \autoref{sec:technologie}).

\subsection{DiagramEdge}
Eine Kante wird mithilfe der Komponente \textit{DiagramEdge} dargestellt. Diese hat Eigenschaften wie 

\begin{itemize}
  \setlength\itemsep{1em}
    \item \textbf{label}: Allfällige Beschriftung der Kante (Links).
    \item \textbf{sourceNodeId}: ID des Ursprungsknotens.
    \item \textbf{destNodeId}: ID des Zielknotens.
    \item \textbf{direction}: Richtung des Links.
\end{itemize}

, welche die Kante näher beschreiben (\autoref{listing:diagramedge}). Die Richtung des \textit{Links} ergibt sich aus der Angabe von \textit{Source-} und \textit{DestinationNodeId}.

\begin{listing}[H]
\inputminted[
frame=lines,
framesep=2mm,
baselinestretch=1.2,
linenos,
breaklines=true
]{js}{sourcecode/common/components/DiagramEdgeInterface.tsx}
\caption{DiamgramEdge-Komponente}
\label{listing:diagramedge}
\end{listing}

\subsection{DiagramNode}
Ähnlich wie die \textit{DiagramEdge} funktioniert die Repräsentation auch bei der \textit{DiagramNode}. Wiederum sind hier Eigenschaften vorgegeben, welche zusätzlich Informationen zum Knoten beinhalten (\autoref{listing:diagramnode}):

\begin{itemize}
    \item \textit{id}: ID
    \item \textit{label}: Beschriftung
    \item \textit{positionX}, \textit{positionY}: Koordinaten
    \item \textit{detailComponenten}: Kind-Komponente für weitere Informationen
\end{itemize}

\begin{listing}[htbp]
\inputminted[
frame=lines,
framesep=2mm,
baselinestretch=1.2,
linenos,
breaklines=true
]{js}{sourcecode/common/components/DiagramNodeInterface.tsx}
\caption{DiamgramNode-Komponente}
\label{listing:diagramnode}
\end{listing}

\subsection{GraphScreen}\label{GraphScreen}
Die \textit{GraphScreen}-Komponente sammelt alle Abhängigkeiten und repräsentiert das Netzwerk als Ganzes. Sie hält alle Referenzen zu Daten, Funktionen sowie zu den umliegenden Komponenten und bildet damit ein wichtiges Bindeglied zwischen \textit{ikc-core} und der Visualisierung. Nachfolgend weitere Informationen zu den Eigenschaften (\autoref{listing:graphscreen}):

\begin{itemize}
    \item \textit{viewToLoad}: Sicht, welche angezeigt wird.
    \item \textit{onViewSave}: Funktion, welche zum Speichern der Sicht aufgerufen wird.
    \item \textit{onViewDelete}: Funktion, welche zum Löschen der Sicht aufgerufen wird.
    \item \textit{nodeComponentFactory}: \textit{Factory}, welche Knoten und Kanten liefert.
    \item \textit{nodeInformationProvider}: Stellt Informationen zu Knoten und Kanten zur Verfügung.
    \item \textit{operationService}: Ermöglicht Interkation mit Datenbasis.
\end{itemize}

\begin{listing}[htbp]
\inputminted[
frame=lines,
framesep=2mm,
baselinestretch=1.2,
linenos,
breaklines=true
]{js}{sourcecode/common/components/GraphScreenInterface.tsx}
\caption{GraphScreen-Komponente}
\label{listing:graphscreen}
\end{listing}

In \autoref{listing:graphscreenexample} ist ein Beispiel für die Verwendung aufgezeigt. 

\begin{listing}[htbp]
\inputminted[
frame=lines,
framesep=2mm,
baselinestretch=1.2,
linenos,
breaklines=true
]{js}{sourcecode/GraphScreenExample.tsx}
\caption{GraphScreen-Beispiel}
\label{listing:graphscreenexample}
\end{listing}

\section{Framework-Auswahl}
Um sicherzustellen, dass ein geeignetes \textit{Framework} für die Visualisierung verwendet wird, müssen verschiedene Optionen untersucht und bewertet werden.
\subsection{Kriterien-Katalog}
Für die Bewertung wird ein Katalog an Kriterien definiert, welcher sich an den definierten Anforderungen (siehe \autoref{anforderungen}), Risikoanalyse (siehe \autoref{risikoanalyse}), Schnittstellen (siehe \autoref{schnittstellen}) und Komponenten (siehe \autoref{komponenten}) orientiert. Folgend eine Übersicht über den Kriterienkatalog (\autoref{tab:kriterien-katalog}). 

\begin{longtable}{|p{1cm}| p{3cm} | p{8.1cm}|}
  \hline
    ID & Titel &  Beschreibung \\\hline
    K1 & Mobile Darstellung & Das Framework kann im Mobile Umfeld verwendet werde\\\hline
    K2 & Operationen & Es können die folgenden Operationen umgesetzt werden:
        \begin{enumerate}
          \item Einen neuen Node innerhalb Visualisierung erstellen.
          \item Zwei Nodes innerhalb Visualisierung verbinden.
          \item Einen Node innerhalb der Visualisierung löschen.
        \end{enumerate} \\\hline
    K3 & Operationen (D'n'D\footnote{Drag and Drop}) & Es können die folgenden D'n'D-Operationen umgesetzt werden:
        \begin{enumerate}
          \item Einen neuen Node mittels Drag'n'Drop innerhalb der Visualisierung erstellen.
          \item Zwei Nodes innerhalb der Visualisierung mittels Drag'n'Drop verbinden.
        \end{enumerate} \\\hline
    K4 & Node Details & Node Details können angezeigt und bearbeitet werden.\\\hline
    K5 & Menu & Ein Kontextmenü für weitere Operationen anzeigen.\\\hline
    K6 & Nodes Speichern & Positionen der Nodes können gespeichert werden.\\\hline
    K7 & Nodes Laden & Eine bestehende View kann dargestellt werden.\\\hline
    K8 & Toolbox & Weitere Operationen, z.b. Suche, können in einer Toolbox angeboten werden.\\\hline
    K9 & Komplexität & Allgemeiner Eindruck des Frameworks hinsichtlich der Komplexität.\\\hline
    K10 & Erweiterbarkeit & Allgemeiner Eindruck des Frameworks hinsichtlich der Erweiterbarkeit.\\\hline
    K11 & Dokumentation & Das Framework ist gut dokumentiert, es gibt genügend Beispiele und eine entsprechende \textit{Community} zur allfälligen Unterstützung.\\\hline
    \caption{Kriterienkatalog}
  \label{tab:kriterien-katalog}
\end{longtable}
Mit Hilfe dieses Katalogs sollen die Möglichkeiten, Chancen und auch Risiken der verschiedenen \textit{Frameworks} identifiziert werden. Aufgrund dieses Prozesses kann anschliessend eine genaue Einschätzung der Möglichkeiten und des Aufwands aufgestellt werden. Die verschiedenen  \textit{Frameworks} werden anhand der folgenden Skala bewertet (1-5): 
\begin{enumerate}
  \item Die Erfüllung des Kriteriums ist mit diesem Framework ohne Anpassungen möglich.
  \item Die Erfüllung des Kriteriums ist mit diesem Framework mit leichten Anpassungen möglich.
  \item Die Erfüllung des Kriteriums ist mit diesem Framework mit Anpassungen möglich.
  \item Die Erfüllung des Kriteriums ist mit diesem Framework mit grossen Anpassungen möglich.
  \item Die Erfüllung des Kriteriums ist mit diesem Framework nicht möglich.
\end{enumerate}

\subsection{Ergebnisse}
Aufgrund des definierten Kriterienkataloges wurden die verschiedenen \textit{Frameworks} untersucht und bewertet. Die Ergebnisse sind in folgender \autoref{tab:framework-auswertung} aufgeführt. Die Tabelle widerspiegelt die Eindrücke und Erfahrungen, welche während den Untersuchen gemacht wurden:

Zwar sind auch die umfassenderen Lösungen sehr interessant, jedoch ist deren Verwendung für den hier notwendigen Zweck zu kompliziert. Es würde lediglich ein kleiner Teilbereich der bereits bestehenden Lösung genutzt. Um aber diesen erfolgreich einzusetzen, ist dennoch tiefe Kenntnis des jeweiligen \textit{Frameworks} erforderlich. Dies sprengt schlichtweg den zeitlichen Rahmen und ist auch nicht notwendig. Die nötigen Erweiterungen können in den anderen, schlankeren \textit{Frameworks} ohne grossen Zusatzaufwand ergänzt werden.

Darum fiel die Wahl eindeutig auf \textit{cytoscape.js}. Es beschränkt sich lediglich auf die Darstellung von Netzwerken. Erweiterungen sind bereits diverse zugänglich. Auch ist es gleichzeitig relativ einfach den Funktionsumfang eigenhändig zu ergänzen.

\begin{longtable}{|p{0.8cm}| p{2.2cm} | p{1.5cm}| p{1.5cm}| p{2cm}| p{2cm}|}
  \hline
    & \textit{cytoscape.js} & \textit{greuler} &\textit{JointJS} &\textit{jsPlumb} &\textit{mxGraph}  \\\hline
    Total & 23 & 27 & 38 & &\\\hline
    \caption{Framework-Auswerung}
  \label{tab:framework-auswertung}
\end{longtable}

\subsection{Bewertungen}\label{Bewertungen}
Eine detaillierte Bewertung der fünf verschiedenen \textit{Frameworks} kann den folgenden Abschnitten entnommen werden.

\subsubsection{cytoscape.js}
\label{cytoscape}
Cytoscape ist eine \textit{open-source} Javascript Bibliothek für Graphen- oder Netzwerk-Theorie. Sie eignet sich nicht nur für Visualisierungen, sondern auch für Analysen. Cytoscape ist mit allen gängigen Browsern und Bibliotheken kompatibel. Auch funktioniert die Darstellung ohne zusätzlichen Aufwand auf allen Bildschirmgrössen. Es gibt zahlreiche Erweiterungen und auch die Integration in bestehende Lösungen ist einfach möglich. \cite{1_franz_lopes_huck_dong_sumer_bader_2016}

Wie in \autoref{fig:bsp-cytoscape} ersichtlich, beschränkt sich die Bibliothek lediglich auf die Visulisierung von Netzwerken. Standardmässig sind keine Zusatzfunktionen, beispielsweise das Interagieren mit \textit{Nodes} und \textit{Links} möglich. Im Gegenzug ist die Bibliothek sehr schlank gehalten, was eine einfache Integration und Erweiterung stark vereinfacht. Das Netzwerk wird direkt im \textit{JSON}-Format hinterlegt. Sollen Änderungen vorgenommen werde, werden die hinterlegten Daten angepasst und anschliessend angezeigt.

\begin{figure}[htbp]
\centering
\includegraphics[width=0.4\textwidth]{cytoscape}
\caption{Beispiel Cytoscape}
\label{fig:bsp-cytoscape}
\end{figure}


\begin{longtable}{|p{0.5cm}|p{0.5cm}|p{0.5cm}|p{0.5cm}|p{0.5cm}|p{0.5cm}|p{0.5cm}|p{0.5cm}|p{0.5cm}|p{0.7cm}|p{0.7cm}|}
  \hline
    K1 & K2 & K3 & K4 & K5 & K6 & K7 & K8 & K9 & K10 & K11 \\\hline
    2 & 2 & 2 & 2 & 2 & 2 & 3 & 2 & 2 & 3 & 1\\\hline
    \caption{Bewertung  \textit{cytoscape.js}}
  \label{tab:bewertung-cytoscape}
\end{longtable}

\subsubsection{greuler}
Hier gibt es viele Parallen zur vorgängigen Bibliothek (\autoref{cytoscape}). \textit{Greuler} spezialisert sich nun aber auf die Visualisierung, baut wie viele andere auf \textit{D3}. Bezüglich der Handhabung ist es grösstenteils identisch mit \textit{Cytoscape}, allerdings hat hier es bei weiten nicht so viele Erweiterungsmöglichkeiten. \autoref{fig:bsp-greuler} zeigt ein kleines Beispielnetzwerk. \cite{2_maurizzzio/greuler_2016}

\begin{figure}[htbp]
\centering
\includegraphics[width=0.4\textwidth]{greuler}
\caption{Beispiel Greuler}
\label{fig:bsp-greuler}
\end{figure}

\begin{longtable}{|p{0.5cm}|p{0.5cm}|p{0.5cm}|p{0.5cm}|p{0.5cm}|p{0.5cm}|p{0.5cm}|p{0.5cm}|p{0.5cm}|p{0.7cm}|p{0.7cm}|}
  \hline
    K1 & K2 & K3 & K4 & K5 & K6 & K7 & K8 & K9 & K10 & K11 \\\hline
    2 & 2 & 2 & 2 & 2 & 4 & 4 & 2 & 2 & 4 & 3\\\hline
    \caption{Bewertung \textit{greuler}}
  \label{tab:bewertung-greuler}
\end{longtable}

\subsubsection{JointJS}
Auch bei JointsJS handelt es sich um eine \textit{open-source}-Bibliothek. Im Gegensatz zu den obigen beiden geht die Funktionalität einen Ebene weiter. Hier werden zusätzlich zur eigentlichen Darstellung auch Werkzeuge zur Manipulation des Diagramms mitgeliefert. Viele Funktionalitäten sind aber erst mit der kostenpflichten Version \textit{Rappid} ver\-füg\-bar, welche auf JointJS aufbaut. \cite{jointsjs}

Die Bibliothek ist für den hier vorliegenden Anwendungsfall zu umfangreich. Die vielen Funktionen machen die Implementation sehr komplex. Dieser Aufwand ist für die eigentliche einfache Anwendung übertrieben. Die Vorteile, welche der vorhandene Funktionsumfang und auch wirklich genutzt werden, können vergleichsweise schnell selbst implementiert werden. So ist die Übersicht stets vorhanden.

\begin{longtable}{|p{0.5cm}|p{0.5cm}|p{0.5cm}|p{0.5cm}|p{0.5cm}|p{0.5cm}|p{0.5cm}|p{0.5cm}|p{0.5cm}|p{0.7cm}|p{0.7cm}|}
  \hline
    K1 & K2 & K3 & K4 & K5 & K6 & K7 & K8 & K9 & K10 & K11 \\\hline
    1 & 3 & 3 & 3 & 3 & 4 & 4 & 5 & 4 & 5 & 3\\\hline
    \caption{Bewertung \textit{JointJS}}
  \label{tab:bewertung-jointjs}
\end{longtable}

\subsubsection{jsPlumb}

\todo[]{Bewertung der Frameworks inkl. kleiner Text + Screenshoot}

\begin{longtable}{|p{0.5cm}|p{0.5cm}|p{0.5cm}|p{0.5cm}|p{0.5cm}|p{0.5cm}|p{0.5cm}|p{0.5cm}|p{0.5cm}|p{0.7cm}|p{0.7cm}|}
  \hline
    K1 & K2 & K3 & K4 & K5 & K6 & K7 & K8 & K9 & K10 & K11 \\\hline
    & & & & & & & & & &\\\hline
    \caption{Bewertung \textit{jsPlumb}}
  \label{tab:bewertung-jsplumb}
\end{longtable}


\subsubsection{mxGraph}

\begin{longtable}{|p{0.5cm}|p{0.5cm}|p{0.5cm}|p{0.5cm}|p{0.5cm}|p{0.5cm}|p{0.5cm}|p{0.5cm}|p{0.5cm}|p{0.7cm}|p{0.7cm}|}
  \hline
    K1 & K2 & K3 & K4 & K5 & K6 & K7 & K8 & K9 & K10 & K11 \\\hline
    & & & & & & & & & &\\\hline
    \caption{Bewertung \textit{mxGraph}}
  \label{tab:bewertung-mxgraph}
\end{longtable}

 
\section{User Interface Design}

Folgend werden wichtige Punkte zum Umgang mit dem Netzwerk und zur Integration der Visualisierung aufzeigt und weitergehende Überlegungen dazu dargelegt.

\subsection{Umgang mit dem Netzwerk}

Die Benutzeroberfläche orientiert sich hauptsächlich am Netzwerk bestehend aus Nodes und Links. Diese besitzen eine Beschriftung in Form des Titels, oder in Form des entsprechenden \textit{Tags} (\autoref{fig:nodes-links}). Diese können zugunsten der Übersichtlichkeit gegebenenfalls ausgeblendet werden.

Wo immer möglich soll die Interaktion nicht über eine entfernte Schaltfläche, sondern direkt am jeweiligen Element in der Visualisierung geschehen. Ein mögliches Beispiel dafür zeigt \autoref{fig:node-menu}. Gleich beim Knoten, wo eine Aktion vorgenommen werden soll, kann mittels eines Klicks (oder Ähnlichem) ein Kontextmenü geöffnet werden. Dies ermöglicht der Situation entsprechende, weiterführende Funktionalitöten. Entsprechende Möglichkeiten sind auch beim Umgang mit Kanten vorstellbar.

\begin{figure}[htbp]
\centering
 \begin{subfigure}{0.35\textwidth}
        \centering
        \includegraphics[width=0.9\linewidth]{skizze-nodes-links}
        \caption{Nodes und Links}
        \label{fig:nodes-links}
    \end{subfigure}
 \begin{subfigure}{0.35\textwidth}
        \centering
        \includegraphics[width=0.9\linewidth]{skizze-contextmenu}
        \caption{Node-Menü}
        \label{fig:node-menu}
    \end{subfigure}
    \caption{Skizzen Benutzeroberfläche}
\end{figure}

\subsection{Integration}

Aus den Anforderungen geht die Interaktion mit dem \textit{ikc-core} als weiterer wichtiger Punkt hervor. Von besonderer Bedeutung ist hier die Bedienung mittels \textit{Drag and Drop}. Wie in \autoref{fig:grenze-core-visual} sichtbar, muss dabei die Grenze zwischen \textit{ikc-core} und der Visualisierung (grüne Linie) überwunden werden. Mittels \textit{Drag and Drop} kann also beispielweise ein Node aus der \textit{ikc-core}-Komponente (orange) in die Visualisierung positioniert werden. Dort kann dieser weiter mit dem bestehenden Netzwerk verknüpft oder als eigenständiger neuer Knoten dargestellt werden.

\begin{figure}[htbp]
\centering
\includegraphics[width=0.5\textwidth]{ikc-core-visual}
\caption{Grenze ikc core - visual}
\label{fig:grenze-core-visual}
\end{figure}

Wo immer möglich sollen bestehende Komponenten wiederverwendet werden. Dies bietet sich nicht nur bei \autoref{fig:node-menu} in Form eines Kontextmenüs, sondern auch für die Suchfunktion und die Detailansicht an. Für die Desktop-Ansicht wird die Visualisierung mittig in einer dreispaltigen Anordnung in den \textit{ikc-core} eingebettet (\autoref{fig:3-column}). Auf der linken Seite befindet sich die Suche. Auf der rechten Seite die Detailansicht für den in der Visualisierung ausgewählten Node. Dort sind auch weitere Optionen beispielsweise in Form von Menüpunkten, den ausgehenden Link erreichbar. Zusätzlich gibt es eine übergeordnete Navigationsleiste, wo wichtige Funktion direkt abrufbar sind.

Bei der Version für Tablets und Smartphones sind, je nach verfügbarer Breite, nur eine oder zwei Spalten sichtbar.


\begin{figure}[htbp]
\centering
\includegraphics[width=0.8\textwidth]{integration-3-column}
\caption{3-spaltige Oberfläche}
\label{fig:3-column}
\end{figure}

%\chapter{Implementation} \label{implementation}

Die Implementation hat zum Ziel, die bereits erarbeitete konzeptionelle Lösung in Form von Code praktisch umzusetzen. Nach einer Übersicht über die abgeschlossene Visualisierung und einem Einblick in die eingesetzten Technologien wird im Detail auf nennenswerte Eigenheiten des Projektes eingegangen.

\section{Übersicht}
Nachfolgend ist die integrierte Visualisierung innerhalb des \gls{ikc-core}[s] ersichtlich (\autoref{fig:overview-implementation}). Sie wurde zwischen der Suche und der Detail-Ansicht platziert. Eine detaillierte Anleitung der Interaktion mit der integrierten Visualisierung ist dem Benutzerhandbuch zu entnehmen. 

\begin{figure}[htbp]
\centerline{\includegraphics[width=1.3\textwidth]{Overview-Screen}}
\caption {Überblick Implementation}
\label{fig:overview-implementation}
\end{figure}

\section{Technologie} \label{sec:technologie}

Für die Umsetzung wurde aufgrund vieler Parallelen zum \gls{ikc-core} und diverser guten Erfahrungen auf ähnliche Technologien gesetzt. Das Grundgerüst ist daher nahezu identisch mit dem bisherigen. Dabei konnte bereits vorhandenes Wissen wiederverwendet und erweitert werden.

\subsection{Typescript}\label{typescript}

Typescript ist eine von Microsoft entwickelte statisch typisierte Erweiterung von Javascript. Durch die Kompilierung wird es zu üblichen Javascript. Typescript kann bereits während des Schreibens überprüft werden und die Entwicklungstools bieten dadurch viel Hilfestellung. Die Typen sind zwar optional, falls eingesetzt bringen sie aber einige hilfreiche Zusatzfunktionen. Beispielsweise lassen sich Klassen, Interfaces definieren. Dies ist vor allem bei der Entwicklung von grösseren Projekten hilfreich. \citep{typescriptlang}

Typescript wurde vor allem aufgrund der ein\-fach\-er\-en Struk\-tu\-rie\-rung von Projekten und der Typisierung eingesetzt. Zwar kann jegliche Ja\-va\-scri\-pt-Bibliothek eingebunden werden, aber ohne bereits vorhandene statische Typisierung in Form von \textit{Typings} bringt dies keine grossen Vorteile. Unglücklicherweise sind die Typisierungen noch nicht für alle Bibliotheken verfügbar und darum muss teilweise auf eine weniger elegante Integration ausgewichen werden.

\subsection{React}\label{react}

React ist eine Bibliothek für die Entwicklung von Benutzeroberflächen, entwickelt von Facebook. Der Fokus wird insbesondere auf die Interaktivität der Komponenten gelegt, deren Zustände ständig wechseln können. Bei Zustandsänderungen aktualisiert React automatisch alle notwendigen Komponenten. Die Komponentenbauweise bringt hohe Modularität und Wiederverwendbarkeit mit sich, untereinander sind die Komponenten entkoppelt. Einmal erstellte Komponenten können an beliebigen Stellen wiederverwendet werden, so zum Beispiel ein Suchfeld. Verwendet werden dies Komponenten immer in der \textit{XML}-Syntax. Grundsätzlich wird eine React-Komponente in verschiedene Teile zerlegt (\autoref{listing:react-component}): 

\begin{enumerate}
    \item \label{props} \textit{Props-Schnittstelle} - definiert, mit welchen Parameter eine Komponente benutzt werden muss. Optionale Parameter werden mit einem Fragezeichen (\texttt{\textbf{?}}) gekennzeichnet. Innerhalb der Komponente kann \texttt{this.props} auf die Eigenschaften zugreifen. Jedoch können keine Werte aktualisiert werden.
    \item \label{states} \textit{State-Schnittstelle} - spezifiziert den Zustand der Komponente. Diese kann sich abhängig von Ereignissen innerhalb der Komponente aktualisieren. Mit \texttt{this.state} kann ein Zugriff stattfinden.
    \item \textit{Implementierung} - jede Komponente implementiert die Schnittstelle \textit{React.Component$<>$}. Einzig die Methode \texttt{render} muss implementiert werden. Darin wird der darzustellende Inhalt aus \gls{HTML}-Elementen oder weiteren React-Komponenten zusammengestellt. 
    \item \textit{Verwendung} - nach der Implementierung kann jede Komponente von anderen React-Komponenten innerhalb deren \texttt{render} Methode verwendet werden.
\end{enumerate}

\begin{listing}[htbp]
\inputminted[
frame=lines,
framesep=2mm,
baselinestretch=1.2,
linenos,
breaklines=true
]{js}{sourcecode/common/react.ts}
\caption{Beispiel React Komponente}
\label{listing:react-component}
\end{listing}

Jede React-Komponente durchläuft grundlegende Zustände. An diesen kann mit sogenannten \textit{Lifecycle}-Methoden in den jeweiligen Prozess eingegriffen werden. Bei der Implementierung der Visualisierung wurden hauptsächlich folgenden Methoden eingesetzt.
%Zusätzlich durchläuft jede React-Komponente drei grundlegende Zustände, in welchen mit verschiedenen speziellen Methode darauf reagiert werden kann:

\begin{itemize}
   \item \textit{Mounting} - hier werden verschiedene Methoden abgearbeitet, bis eine Komponente erstellt und zu einem \gls{HTML}-Element konvertiert wird.
   \begin{enumerate}
     \item \texttt{constructor}
     \item \texttt{componentWillMount}
     \item \texttt{render}
     \item \texttt{componentDidMount}
   \end{enumerate}
   \item \textit{Updating} - ein Update kann entweder durch eine Aktualisierung des \textit{State} oder der \textit{Props} erfolgen.
   \begin{enumerate}
     \item \texttt{componentWillReceiveProps}
     \item \texttt{shouldComponentUpdate}
     \item \texttt{componentWillUpdate}
     \item \texttt{render}
     \item \texttt{componentDidUpdate}
   \end{enumerate}
   \item \textit{Unmounting} - dieser Schritt wird ausgeführt, bevor die Komponente entfernt wird.
   \begin{enumerate}
     \item \texttt{componentWillUnmount}
   \end{enumerate}
\end{itemize}

Ein weiterer, besonderer Punkt von React ist der uni\-di\-rek\-tio\-nale Datenfluss. Dieser wird nachfolgend im \autoref{unidirectional} behandelt. \citep{reactjs, reactjs-blog}

\subsection[Datenfluss]{Unidirektionaler Datenfluss}\label{unidirectional}
Die Verwendung von unidirektionalen Datenflüssen erleichtert die Verwendung von React-Komponenten, welche auch in dem, von Facebook präsentierten, Architektur-Muster \textit{Flux} enthalten sind. Auf eine strikte Verwendung von \textit{Flux} wurde verzichtet, jedoch wurden trotzdem unidirektionale Datenflüsse eingesetzt. Wie diese genau funktionieren, wird anhand des Beispiels in \autoref{fig:unidirectional} genauer aufgezeigt: 
   \begin{enumerate}
     \item Die \textit{GraphScreen}-Komponente verwendet die \textit{Graph}-Komponente in ihrer \texttt{render}-Methode in Form von (\texttt{<Graph ... />}). 
     \item Sobald den \textit{Graph}-Komponente erstellt ist, wird der Zustand anhand der übergebenen Parameter initialisiert.
     \item Angestossen von \gls{Event}s aus dem \textit{cytoscape}-Framework werden \gls{Callback}-Methoden aufgerufen, um Information an die \textit{GraphScreen}-Komponente zu senden. Die entsprechenden Methoden werden als Parameter an die \textit{Graph}-Komponente übergeben, beispielsweise \texttt{this.prop.handleNewLink(...)}
     \item Innerhalb den von der \textit{Graph}-Komponente aufgerufenen Methoden in der \textit{GraphScreen}-Komponente wird nun der Zustand der \textit{GraphScreen}-Komponente aktualisiert. Dies geschieht mit der Methode \texttt{this.setState({...})}, welche von dem React-Framework zur Verfügung gestellt wird. Hier werden die Änderungen weitergegeben. Diese werden im Hintergrund ausgeführt.
     \item Sobald der Zustand angepasst wurde, wird die \texttt{render}-Methode neu ausgeführt. Dadurch werden die Aktualisierungen der \textit{Graph}-Komponente per Parameter (\textit{Props}) weitergegeben.
     \item Durch die Aktualisierung der Parameter wird nun auch die \texttt{render} Methode der \textit{Graph}-Komponente neu ausgeführt und der Zustand aktualisiert. 
   \end{enumerate}

\begin{figure}[htbp]
\centerline{\includegraphics[width=1\textwidth]{unidirectional}}
\caption {Unidirektionaler Datenfluss}
\label{fig:unidirectional}
\end{figure}


\subsection{Material Design}

Die React Erweiterung \textit{Material UI} ist zuständig für das Erscheinungsbild der Visualisierung, wie auch schon beim \textit{ikc-core}. Dafür gibt es diverse vorgefertigte Komponenten, welche praktisch ohne Zusatzaufwand verwendet werden können. Googles Material Design kombiniert klassische Design Prinzipien mit Innovation auf Technik und Wissenschaft. \citep{react-material-ui, google-material-ui}

\subsection{Webpack}

Mit der Verwendung von zahlreichen Erweiterungen und \hyperref[typescript]{\textit{Typescript}} als primäre Programmiersprache ergeben sich diverse Abhängigkeiten und auch eine Kompilierung ist notwendig. Für die Sammlung der Abhängigkeiten und die Übersetzung in Javascript wird Webpack eingesetzt. Das Resultat ist beispielweise eine Javascript-Datei, welche alle Abhängigkeiten im richtigen Format enthält. Darum ist vom Webserver nur eine Datei zu laden, so verkürzt die Ladezeit zusätzlich. \citep{webpack}

\section[Interkation]{Interaktion mit bestehenden Komponenten}\label{interaktion}

Innerhalb der Visualisierung werden verschiedene bestehende Komponenten aus dem \gls{ikc-core} genutzt. Diese implementieren die beschriebenen Schnittstellen (\autoref{schnittstellen}) und werden an den \textit{GraphScreen} üb\-er\-ge\-ben. Die Implementationen gelten als Voraussetzung für die Verwendung der Visualisierung.

\subsubsection{GraphDialogFactory/GraphSearchFieldFactory}
Mit Hilfe dieser beiden Klassen kann die Visualisierung auf \hyperref[react]{\textit{React}}-Komponenten zugreifen, welche im \gls{ikc-core} implementiert sind. Diese sind in erster Linie Dialoge oder Suchfelder. Im folgenden Ablaufdiagramm (\autoref{fig:interaction-dialog} und \autoref{listing:interaction-factory}) ist die Funktionsweise am Beispiel des \textit{NewNodeDialog}s detailliert ersichtlich. Das Prinzip gilt ebenfalls für die Dialoge \textit{NewNodeToConnect}, \textit{NodeSearchToConnect} und die beiden Suchfelder \textit{NodeSearchField} und \textit{LinkSearchField}:

\begin{enumerate}
    \item Die Komponente \textit{GraphVisualisation} erstellt ein Objekt der Klasse \textit{GraphDialogFactory}. Sie implementiert die Schnittstelle \hyperref[DialogFactory]{\textit{DialogFactory}} aus der Visualisierung.
    \item Bei der Verwendung der Visualisierung (\textit{GraphScreen}) wird unter anderem das \textit{GraphDialogFactory}-Objekt an die Visualisierung übergeben. Dabei ändert das Objekt den Typ von \textit{GraphDialogFactory} zu \hyperref[DialogFactory]{\textit{DialogFactory}}. Da ersteres der Visualisierung unbekannt ist, kann so eine beidseitige Kopplung verhindert werden. Alle Objekte, welche übergeben werden, sind in der Schnittstelle \hyperref[GraphScreenProps]{\textit{GraphScreenProps}} zusammengefasst. Diese wird innerhalb des \hyperref[react]{\textit{React}}-Frameworks implementiert.
    \item Wenn nun die Visualisierung einen \textit{NewNode}-Dialog benötigt, wird dieser von der \textit{GraphDialogFactory} bezogen. Diese wird von dem \hyperref[GraphScreenProps]{\textit{GraphScreenProps}} zur Verfügung gestellt. Dazu wird die Methode \texttt{getNewNodeDialog} aufgerufen, welcher die benötigten Informationen für die Erstellung des Dialogs mitgegeben werden.
    \item Nachdem der Dialog geschlossen wurde, werden die entsprechenden \gls{Callback}-Methoden ausgeführt. Diese stehen dem Dialog durch das \textit{DialogNewNodeProps}-Objekt zu Verfügung, welches zuvor von der \textit{GraphDialogFactory} anhand der Informationen der Visualisierung entsprechend bestückt wurde. Somit wird die Visualisierung über das Resultat des Dialogs informiert und kann die Informationen entsprechend weiterverarbeiten. 
\end{enumerate}

\begin{figure}[htbp]
\centerline{\includegraphics[width=1.3\textwidth]{ExternalComponentDialog}}
\caption{Interaktion mit den Factories}
\label{fig:interaction-dialog}
\end{figure}

\begin{listing}[htbp]
\inputminted[
frame=lines,
framesep=2mm,
baselinestretch=1.2,
linenos,
breaklines=true
]{js}{sourcecode/common/InteractionFactory.ts}
\caption{Beispiel Interaktion Factory}
\label{listing:interaction-factory}
\end{listing}

\subsubsection{Weitere Implementationen}
Mit Hilfe der Implantation der Schnittstellen \hyperref[NodeInformationProvider]{\textit{NodeInformationProvider}},  \hyperref[OperationService]\textit{OperationService},  \hyperref[IdentityService]{\textit{IdentityService}} können der Visualisierung konkrete Implementationen zu Verfügung gestellt werden (\autoref{fig:interaction-nodeinformationprovider}, \autoref{fig:interaction-identityservice}, \autoref{fig:interaction-operationservice} und \autoref{listing:interaction-components}):
\begin{enumerate}
    \item Das entsprechende Objekt der Implementierung wird erstellt.
    \item Bei der Erstellung der Visualisierung wird das Objekt übergeben.
    \item Über \hyperref[GraphScreenProps]{\textit{GraphScreenProps}} kann die Visualisierung auf das entsprechende Objekt zugreifen und mit den Methoden die benötigten Informationen abrufen.
\end{enumerate}

\begin{figure}[htbp]
\centerline{\includegraphics[width=1.3\textwidth]{ExternalComponentInformationProvider}}
\caption{Interaktion mit dem NodeInformationProvider}
\label{fig:interaction-nodeinformationprovider}
\end{figure}


\begin{figure}[htbp]
\centerline{\includegraphics[width=1.3\textwidth]{ExternalComponentIdentityService}}
\caption{Interaktion mit dem IdentityService}
\label{fig:interaction-identityservice}
\end{figure}


\begin{figure}[htbp]
\centerline{\includegraphics[width=1.3\textwidth]{ExternalComponentOperationService}}
\caption{Interaktion mit dem OperationService}
\label{fig:interaction-operationservice}
\end{figure}

\begin{listing}[htbp]
\inputminted[
frame=lines,
framesep=2mm,
baselinestretch=1.2,
linenos,
breaklines=true
]{js}{sourcecode/common/InteractionComponents.ts}
\caption{Beispiel: Interaktion weiterer Komponenten}
\label{listing:interaction-components}
\end{listing}

\section{Drag'n'Drop}
\label{dnd}
Die \gls{Drag'n'Drop}-Funktionalität bietet sowohl auf dem mobilen als auch auf dem Desktop-Gerät die Möglichkeit, Benutzerinteraktion intuitiv und einfach zu gestalten. Wie im \autoref{subsec:aktionen} bereits definiert, sind dies \textit{Add Node}, \textit{New Link}, \textit{Link To Existing Node} und \textit{Update Position}. Dies kann sowohl innerhalb der Visualisierung oder auch von einer externen Komponente aus geschehen. In diesem Abschnitt werden die grundsätzlichen Abläufe erläutert, welche es ermöglichen einen \gls{Node} von einer externen Komponente in die Visualisierung zu ziehen. Wie die Aktionen innerhalb der Visualisierung funktionieren, wird im \autoref{subsec:graph-manipulation} erläutert.

Grundsätzlich wird der Vorgang des \gls{Drag'n'Drop}[s] in vier Phasen unterteilt:

\begin{itemize}
    \item \textit{Registration} - Alle Elemente, auf welche ein Drag (Nodes) oder ein Drop (Visualisierung) möglich sein soll, müssen registriert werden. 
    \item \textit{Drag} - Ein Drag wird mit einem Klick oder Tap auf einem Element gestartet.
    \item \textit{Move} - Das Element wird auf dem Monitor bewegt.
    \item \textit{Drop} - Das Element wird auf der Visualisierung losgelassen. Wenn dies an einer freien Stelle geschieht, wird der \gls{Node} an dieser Stelle hinzugefügt (4.a). Geschieht dies jedoch über einem bestehenden \gls{Node}, wird ein neuer \gls{Link} erstellt und der neue \gls{Node} im Umkreis des bestehenden positioniert (4.b).
\end{itemize}


\begin{figure}[htbp]
\centerline{\includegraphics[width=1.3\textwidth]{DragNDropScreen}}
\caption{Übersicht \gls{Drag'n'Drop}}
\label{fig:dnd-screen}
\end{figure}

Damit diese Phasen korrekt ausgeführt werden können, ist es wichtig zu wissen, welcher \gls{Node} an der Aktion beteiligt ist. Weiter müssen verschiedene \gls{Event}s abgefangen und verarbeitet werden. Um dies möglichst simpel und effizient zu implementieren, wird dies in eine externe \gls{Javascript}-Datei ausgelagert und somit vom Rest entkoppelt (\autoref{listing:dnd-handling}). Dieses bietet zwei zwingende und zwei optionale Methoden:
\begin{itemize}
    \item \textit{registerDragZone} - Mit der Registration wird das \gls{Drag'n'Drop} auf dem entsprechenden \gls{HTML}-Element aktiviert. Dazu muss das entsprechende \gls{HTML}-Objekt als auch die entsprechende Node-ID übergeben werden.
    \item \textit{registerDropZone} - Ebenfalls muss die Visualisierung als Drop-Zone registriert werden. Dadurch kann bei einem Drop über ihr die \gls{Callback}-Methode \textit{onDrop} ausgeführt werden, welche übergeben werden muss. 
    \item \textit{registerCallbackForStart} - Hiermit wird jedem Element ermöglicht, bei einem Start eines \gls{Drag'n'Drop} informiert zu werden. 
    \item \textit{registerCallbackForEnd} - Dadurch ist es möglich benachrichtigt zu werden, sobald der \gls{Drag'n'Drop}-Vorgang beendet ist.
\end{itemize}

\begin{listing}[htbp]
\inputminted[
frame=lines,
framesep=2mm,
baselinestretch=1.2,
linenos,
breaklines=true
]{js}{sourcecode/common/MobileDrop.js}
\caption{Drag'n'Drop Handhabung}
\label{listing:dnd-handling}
\end{listing}

Wie diese Phasen im Detail abgehandelt wird, ist in Ablaufdiagramm (\autoref{fig:dnd-sequence}) ersichtlich. Sie sind ebenfalls in die vier Phasen des \gls{Drag'n'Drop} unterteilt.

\begin{figure}[htbp]
\centerline{\includegraphics[width=1.2\textwidth]{DragAndDrop}}
\caption{Ablauf \gls{Drag'n'Drop}}
\label{fig:dnd-sequence}
\end{figure}

\section{Visualisierung}\label{visual}
Die Funktionalitäten der Visualisierung werden hauptsächlich durch vier \hyperref[react]{\textit{React}}-Komponenten abgedeckt: \textit{GraphScreen} fungiert dabei als zentrale Stelle und initialisiert alle anderen Komponenten. Insbesondere wird das entsprechende \hyperref[view]{\textit{View}}-Objekt, \textit{viewToLoad} analysiert, die darzustellenden \gls{Node}[s] und \gls{Link}[s] extrahiert und der \textit{Graph}-Komp\-on\-en\-te übergeben.

Sobald innerhalb der Visualisierung ein \gls{Event} aufgrund einer Benutzerinterkation ausgelöst wird, werden die entsprechenden \gls{Callback}-Methoden des \textit{GraphScreen} aufgerufen. Darüber werden die entsprechenden Schritte ausgeführt, welche wiederum zu Aktualisierungen in den einzelnen Komponenten führen (\autoref{fig:component-visualisation}). Jegliche Kommunikation zwischen den Komponenten wird nach dem Prinzip des unidirektionalen Datenflusses (\autoref{unidirectional}) realisiert.

Zum Beispiel soll ein \gls{Link} dargestellt werden. Dazu wird im \textit{GraphScreen} beim entsprechenden \hyperref[GraphLinkElement]{Link} die Eigenschaft \textit{visible} auf \texttt{VISI\-BILI\-TY.VISI\-BLE} gesetzt. Durch die Aktualisierung des Zustands werden nun die \gls{Node}[s] und \gls{Link}[s], welche der \textit{Graph}-Komponente übergeben werden, aktualisiert. Darin enthalten ist nun auch der neu darzustellende Link.

\begin{figure}[htbp]
\centerline{\includegraphics[width=0.5\textwidth]{VisualisationComponents}}
\caption{Komponenten der Visualisierung}
\label{fig:component-visualisation}
\end{figure}


\subsection{Manipulation des \gls{Netzwerk}[s]}
\label{subsec:graph-manipulation}

Die Manipulationen innerhalb des \gls{Netzwerk}[s] haben ihren Ursprung immer im \textit{cytoscape}-Framework. Hier werden verschiedene \gls{Event}s registriert, um entsprechend reagieren zu können. Dies geschieht auf verschiedenen Ebenen der Visualisierung:
\begin{enumerate}
    \item Auf der ganzen Visualisierung, zum Beispiel \textit{Rechtsklick} an einer freien Stelle.
    \item Auf mehreren \gls{Node}[s] der Visualisierung, zum Beispiel \textit{durch Auswahl}.
    \item Auf mehreren \gls{Link}[s] der Visualisierung, zum Beispiel durch \textit{Auswahl}.
    \item Auf einem spezifischen Element der Visualisierung, zum Beispiel \textit{Loslassen} eines \gls{Node}[s].
\end{enumerate}


\begin{listing}[htbp]
\inputminted[
frame=lines,
framesep=2mm,
baselinestretch=1.2,
linenos,
breaklines=true
]{js}{sourcecode/common/cytoscape-event.ts}
\caption{Cytoscape Event Beispiel}
\label{listing:cytoscape-event}
\end{listing}

\subsubsection{Update Position/New Link (Drag'n'Drop)}
Wenn die Aktion \textit{Update Position} und \textit{New Link} per \gls{Drag'n'Drop} ausgeführt werden, startet ein teilweise ähnlicher Ablauf (\autoref{fig:sequence-movenode}):
\begin{enumerate}
    \item Der erste Schritt ist Teil des allgemeinen Ablaufs der Visualisierung. Darin wird die \textit{Graph}-Komponente erstellt. Falls eine Aktion auf einem \gls{Node} ausgeführt wird, wird der \gls{Event} \textit{grab} registriert.
    \item Bei der Auswahl und Bewegung eines \gls{Node}[s], wird nun der \gls{Event} \textit{free} temporär registriert, welcher ausgeführt wird sobald der \gls{Node} losgelassen wird.
    \item Sobald der \gls{Node} losgelassen wird, wird überprüft, ob an dieser Stelle in der Visualisierung ein anderer \gls{Node} positioniert ist. Ist dies der Fall, wird ein neuer \gls{Link} zwischen den zwei \gls{Node}[s] erstellt und dazu folgende Schritte ausgeführt:  
        \begin{itemize}
            \item Die \textit{GraphScreen}-Komponente wird mittels einer \textit{Callback}-Meth\-ode informiert.
            \item Innerhalb der \textit{GraphScreen}-Komponente wird der neue \gls{Link} erstellt und die Datenbasis durch den \hyperref[OperationService]{\textit{OperationService}} informiert.
            \item Nun wird der Zustand der \textit{GraphScreen}-Komponente aktualisiert, die \texttt{render}-Methode ausgeführt und die Aktualisierungen an die \textit{Graph}-Komponente weitergegeben. Zum Schluss werden die Änderungen angezeigt.    
        \end{itemize}
    Wenn dies jedoch nicht zutrifft, werden folgende Schritte ausgeführt. So kann die neue Position des \gls{Node}[s] gespeichert werden: 
        \begin{itemize}
            \item Durch die entsprechende \textit{Callback}-Methode wird die \textit{GraphScreen}-Komponente informiert.
            \item Die Position wird innerhalb der \textit{GraphScreen}-Komponente aktualisiert und die Datenbasis durch den \hyperref[OperationService]{\textit{OperationService}} informiert.
            \item Ebenfalls wird der Zustand der \textit{GraphScreen}-Komponente aktualisiert, die \texttt{render} Methode ausgeführt und die Aktualisierungen an die \textit{Graph}-Komponente weitergegeben. Wiederum werden zum Schluss die Änderungen sichtbar.
        \end{itemize}
\end{enumerate}

\begin{figure}[htbp]
\centerline{\includegraphics[width=1.2\textwidth]{UpdatePositionDiagramm}}
\caption{Ablauf Update Position / New Link}
\label{fig:sequence-movenode}
\end{figure}

\subsubsection{Add Node/Link To Existing \gls{Node} (Drag'n'Drop)}
Im \autoref{dnd} wurde bereits erläutert, wie ein \gls{Drag'n'Drop} eines \gls{Node}[s] von ausserhalb behandelt wird. Dadurch wird es ermöglicht, die Aktionen \textit{AddNode} und \textit{LinkToExistingNode} per \textit{Drag'n'Drop} auszuführen. Dazu werden weitere Schritte ausgeführt, welche auf diejenigen in der \autoref{fig:dnd-sequence} folgen (\autoref{fig:sequence-afterdrop}):
\begin{enumerate}
    \item Auch bei diesen beiden Aktionen werden zuerst die allgemeinen Schritte abgearbeitet und die Visualisierung als \textit{DropZone} registriert (vgl. \autoref{listing:dnd-handling}).  
    \item Sobald nun ein \gls{Node} über der Visualisierung losgelassen wird, wird die \textit{Graph}-Komponente informiert. 
    \item Anhand der Position des losgelassenen \gls{Node}[s] wird nun ermittelt, ob an dieser Stelle bereits ein \gls{Node} in der Visualisierung dargestellt wird. Falls dies zutrifft, muss der losgelassene \gls{Node} in der Visualisierung dargestellt werden und mit einem \gls{Link} zum \gls{Node} an der losgelassenen Stelle verbunden werden. Dafür sind folgenden Schritte notwendig:
        \begin{itemize}
            \item Mithilfe der entsprechenden \textit{Callback}-Methode wird die \textit{GraphScreen}-Komponente informiert.
            \item Die Änderungen werden nun in der \textit{GraphScreen}-om\-po\-nen\-te ausgeführt und die Datenbasis durch den \hyperref[OperationService]{\textit{OperationService}} informiert.
            \item Durch die Aktualisierung des Zustands der \textit{GraphScreen}-Komponente wird die \texttt{render}-Methode ausgeführt und die Aktualisierungen an die \textit{Graph}-Komponente weitergegeben und somit die Änderungen angezeigt.    
        \end{itemize}
    Wenn nun der \gls{Node} an einer freien Stelle losgelassen wird, soll dieser ohne einen neuen \gls{Link} erstellt werden. Dazu sind folgende Schritte nötig:
        \begin{itemize}
            \item Durch die entsprechenden \textit{Callback}-Methode wird die Information an die \textit{GraphScreen}-Komponente weitergegeben.
            \item Änderungen werden nun innerhalb der \textit{GraphScreen}-Kom\-po\-nen\-te abgearbeitet und die Informationen durch den \hyperref[OperationService]{\textit{OperationService}} an die Datenbasis weitergegeben.
            \item Auch hier wird zum Schluss die Aktualisierung des Zustands der \textit{GraphScreen}-Komponente ausgeführt und somit die \texttt{render}-Methode aufgerufen. Die Aktualisierungen werden an die \textit{Graph}-Komponente weitergegeben und somit die Änderungen angezeigt.    
        \end{itemize}
\end{enumerate}
\begin{figure}[htbp]
\centerline{\includegraphics[width=1.2\textwidth]{FurtherProcessDropNode}}
\caption{Ablauf Add Node / Link To Existing Node}
\label{fig:sequence-afterdrop}
\end{figure}


\subsection{Kontextmenüs}
Die beiden Kontextmenüs (\autoref{fig:core-contextmenu} und \autoref{fig:node-contextmenu}) bieten eine Sammlung von \hyperref[subsec:aktionen]{\textit{Aktionen}}, welche der entsprechenden Situation angepasst sind. Beide werden entweder durch einen Rechtsklick (Desktop) oder einen langen Tap (Mobile) aufgerufen. Diese \gls{Event}s werden ebenfalls, wie in \autoref{visual} beschrieben, innerhalb der \textit{Graph}-Komponente beim \textit{cytoscape}-Framework registriert. Sobald der \gls{Event} auftritt werden durch die entsprechenden \textit{Callback}-Methoden die \textit{GraphScreen}-Komponente informiert, damit diese das entsprechende Menü öffnet.

Um die entsprechenden Suchfelder oder Dialoge aufzurufen, nutzen beide Menüs die Implementierung der Schnittstellen \hyperref[SearchFieldFactory]{SearchFieldFactory} oder \hyperref[DialogFactory]{DialogFactory} (\autoref{interaktion}).

\begin{figure}
\centering
\begin{subfigure}[b]{0.5\textwidth}
    \includegraphics[width=1\linewidth]{CoreContextMenu-Screen}
    \caption{CoreContextMenu}
    \label{fig:core-contextmenu}
    \end{subfigure}
\begin{subfigure}[b]{0.4\textwidth}
    \includegraphics[width=1\linewidth]{NodeContextMenu-Screen}
    \caption{NodeContextMenu}
    \label{fig:node-contextmenu}
    \end{subfigure}
    \caption{Kontextmenüs}
\end{figure}


\subsubsection{CoreContextMenu}
Im \textit{CoreContextMenu} (\autoref{fig:core-contextmenu}) werden die beiden Aktionen \textit{AddNode} und \textit{NewNode} zur Verfügung gestellt (\autoref{subsec:aktionen}). Dabei werden durch die Implementierungen der beiden Schnittstellen \hyperref[SearchFieldFactory]{SearchFieldFactory} und \hyperref[DialogFactory]{DialogFactory} das Suchfeld \textit{GraphNodeSearchField} und der Dialog \textit{NewNodeDialog} bezogen und eingebunden (\autoref{fig:sequence-corecontextmenu}).

\begin{figure}[htbp]
\centerline{\includegraphics[width=1.2\textwidth]{CoreContextMenuSequence}}
\caption{Ablauf CoreContextMenu}
\label{fig:sequence-corecontextmenu}
\end{figure}


\subsubsection{NodeContextMenu}
Diverse \gls{Node} spezifische Aktionen werden im \textit{CoreContextMenu} zusammengefasst (\autoref{subsec:aktionen}). Dazu werden ebenfalls mithilfe der Implementierungen der beiden Schnittstellen \hyperref[SearchFieldFactory]{SearchFieldFactory} und \hyperref[DialogFactory]{DialogFactory} das Suchfeld \textit{GraphLinkSearchField} und die beiden Dialoge \textit{NewNodeToConnect} und \textit{ExistingNodeToConnect} genutzt (\autoref{fig:sequence-nodecontextmenu}). 
\begin{figure}[htbp]
\centerline{\includegraphics[width=1.15\textwidth]{NodeContextMenuSequence}}
\caption{Ablauf NodeContextMenu}
\label{fig:sequence-nodecontextmenu}
\end{figure}

\section{Integration}
Mit der Integration der Visualisierung in den \textit{ikc-core} wird die Implementation abgeschlossen. Dazu sind die nötigen Schnittstellen zu implementieren und zusätzlich weitere Anpassungen  (\autoref{fig:integration-visualisation}) notwendig:

\begin{itemize}
    \item \textit{GraphVisualisation} - nutzt die \textit{GraphScreen}-Komponente und üb\-er\-gi\-bt ihr alle nötigen Implementationen und Informationen. 
    \item \textit{GraphNodeInformationProvider} - ermöglicht der Visualisierung den Zugriff auf die Datenbasis, um Informationen abzufragen und implementiert die Schnittstelle \hyperref[NodeInformationProvider]{NodeInformationProvider}.
    \item \textit{GraphOperationService} - Dadurch kann die Visualisierung Informationen an die Datenbasis (\textit{NodeService}, \textit{PropertyService} und \textit{ViewService}) weiterleiten. Dazu wird die Schnittstelle \hyperref[OperationService]{OperationService} verwendet.
    \item \textit{GraphDialogFactory} - kapselt die Erstellung der verschiedene Dialog und stellt diese zur Verfügung. Es werden die verschiedenen Methoden der Schnittstelle \hyperref[DialogFactory]{DialogFactory} implementiert.
    \item \textit{GraphSearchFieldFactory} - um die beiden Suchfelder \textit{GraphNodeSearchField} und \textit{GraphLinkSearchField} der Visualisierung zu Ver\-füg\-ung zu stellen, wird die Schnittstelle \hyperref[SearchFieldFactory]{SearchFieldFactory} implementiert.
    \item \textit{ElementIdentityService} - damit die Visualisierung konsistente IDs für neue \gls{Node}[s] und \gls{Link}[s] vergeben kann, werden diese mithilfe der Schnittstelle \hyperref[IdentityService]{IdentityService} implementiert.
    \item \textit{Dialoge} - die verschiedenen Dialoge, welche der Visualisierung durch die \textit{GraphDialogFactory} zu Verfügung gestellt werden, nutzen die \hyperref[subsec:dialoginterfaces]{Dialog-Schnittstellen}. Diese dienen als Grundlage für die \gls{Props}- und \gls{State}-Schnittstellen der \hyperref[react]{\textit{React}}-Komponente.
    \item Um die verschiedenen \gls{View}s strukturiert halten zu können, werden die Klassen \textit{ViewService} und \textit{ViewModel} verwendet. Innerhalb des \textit{ViewModel}s wird auch die Verbindung zur externen Persistierung auf \gls{Dropbox} implementiert.
    \item Bisher wurde die Applikation nach folgendem Schema aufgerufen: \url{https://<host>/node/:node}, wobei \textit{:node} die darzustellende Node-ID enthält, z.B. \url{https://localhost:8888/node/0}. Neu wird das folgende Schema verwendet: \url{https://<host>/:node(/:view)(/:focus)}. Dabei wurden die beiden optionalen Parameter \textit{:view} und \textit{:focus} hinzugefügt. \textit{:view} enthält die entsprechende ID der View, welche darzustellen ist und \textit{:focus} den Wert \textit{node} oder \textit{view} je nachdem, auf welcher Darstellung der Fokus liegt. Dies ist jedoch nur auf der Mobile-Ansicht relevant. Mit dem Aufruf \url{https://localhost:8888/0/0/view} wird der \gls{Node} mit der ID \textit{0} und die Sicht mit der ID \textit{0} geladen, weiter soll der Fokus auf der Visualisierung liegen, falls die Anfrage von einem mobilen Gerät kommt.
    \item Der mobilen Navigationsleiste wurde ein neues Icon hinzugefügt, mit welchem zwischen der Visualisierung und der NodeDetail-Ansicht gewechselt werden kann. 
    \item Für die Navigation in der Mobile- und der Desktop-Ansicht wurde ein neuer Punkt \textit{View} hinzugefügt. Dadurch können neue Views erstellt, bestehende durchsucht und dargestellt werden. Weiter wurde dem Punkt \textit{Delete} zwei Menüpunkte \textit{DeleteView} und \textit{DeleteNode} hinzugefügt. 
    \item Innerhalb des \textit{NodeService} wurden entsprechende Methoden des \textit{ViewService} eingebaut. Somit wird sichergestellt, dass alle Änd\-er\-ung\-en der Datenbasis auch den verschiedenen Sichten weitergegeben werden. So zum Beispiel falls ein \gls{Link} in der Datenbasis gelöscht wird. Diese Änderung muss auch in den Views ausgeführt werden, sodass dieser \gls{Link} nicht mehr dargestellt wird. 
    
\end{itemize}

\begin{figure}[htbp]
\centerline{\includegraphics[width=1.3\textwidth]{ikc-visual-integration}}
\caption{Intergration Visualisierung}
\label{fig:integration-visualisation}
\end{figure}



%\chapter{Schlussfolgerungen}

\section{Erkenntisse}



\section{Lessons learned}



\section{Ausblick}

\bibliographystyle{plain}
\bibliography{refs}

%\appendix

%\chapter{Appendix}

\section{Testfälle} \label{tests}

\begin{longtable}{|p{1cm} | p{10cm} |p{1.2cm} |}
  \hline
    ID & Ablauf & Erfüllt \\\hline
    T1 & Das (allgemeine) Kontextmenü öffnen und im Untermenü \textit{Add New} den entsprechenden Typ auswählen. Danach öffnet sich ein Dialog, wo der neue \gls{Node} beschrieben werden kann. Nach dem Speichern muss er an der Stelle, wo das Kontextmenü aufgerufen worden ist, angezeigt werden. & Ja \\\hline
    T2 & Das \gls{Node}-Kontextmenü öffnen und im Untermenü \textit{Link To} den entsprechenden Typ auswählen. Danach öffnet sich ein Dialog, wo der neue \gls{Node} beschrieben werden kann. Nach dem Speichern muss der Node, bei welchem das Kontextmenü aufgerufen wurde, mit einem Link zum neuen Node verbunden sein und in der Umgebung darstellt werden. & Ja \\\hline
    T3 & Das (allgemeine) Kontextmenü öffnen und im unteren Bereich den darzustellenden \gls{Node} suchen und auswählen. Danach muss er an der Stelle, wo das Kontextmenü aufgerufen worden ist, zusammen mit seinen Kind-\gls{Node}[s] und den verbindenden Links ersichtlich sein. & Ja\\\hline
    T4 & Das \gls{Node}-Kontextmenü öffnen und im Untermenü \textit{Link To} \textit{Existing Node} auswählen. Danach öffnet sich ein Dialog, wo der bestehende \gls{Node} gesucht und ausgewählt werden kann. Nach dem Speichern muss dieser mit dem Node, auf dem das Kontextmenü aufgerufen wurde, mit einem Link verbunden sein und in der Umgebung dargestellt werden. & Ja\\\hline
    T5 & Aus der Suche einen \gls{Node} per \gls{Drag'n'Drop}  in die Visualisierung ziehen und an einer freien Stelle loslassen. An dieser Stelle muss nun dieser \gls{Node} dargestellt werden. Für die mobile Ansicht muss zuerst die Suche über das entsprechende Icon geöffnet werden. Sobald der \gls{Drag'n'Drop} Prozess beginnt, verschwindet das Suchfeld, um den \gls{Node} optimal positionieren zu können, nach dem Loslassen wird es wieder dargestellt. & Ja \\\hline
    T6 & Aus der Suche einen \gls{Node} per \gls{Drag'n'Drop} in die Visualisierung ziehen und über einem bereits dargestellten \gls{Node} loslassen. Nun muss dieser \gls{Node} mit dem ausgewählten \gls{Node} per Link verbunden werden. Für die mobile Ansicht muss zuerst die Suche über das entsprechende Icon geöffnet werden. Sobald der \gls{Drag'n'Drop}-Prozess beginnt, verschwindet das Suchfeld, um den \gls{Node} optimal positionieren zu können, nach dem Loslassen wird es wieder dargestellt. & Ja \\\hline
    T7 & Das \gls{Node}-Kontextmenü öffnen und mit \textit{Hide Node} den entsprechenden \gls{Node} ausblenden. Dabei müssen alle ein- und ausgehenden Links ebenfalls ausgeblendet werden. & Ja\\\hline
    T8 & Das \gls{Node}-Kontextmenü öffnen und mit \textit{CollapseAllLinks} alle ausgehenden Links ausblenden. Falls der Ziel-\gls{Node} des jeweiligen \gls{Node}[s] keinen anderen ein- oder ausgehenden Link besitzt, muss dieser auch ausgeblendet werden. & Ja\\\hline
    T9 & Das \gls{Node}-Kontextmenü öffnen und mit \textit{ExpandAllLinks} alle ausgehenden Links einblenden. Falls der Ziel-\gls{Node} nicht bereits eingeblendet ist, muss dieser zusammen mit dem Link eingeblendet werden. & Ja\\\hline
    T10 & Das \gls{Node}-Kontextmenü öffnen und im unterem Bereich einen ausgehenden Link auswählen, dieser muss anschliessend zusammen mit dem Ziel \gls{Node}, falls dieser noch nicht dargestellt ist, dargestellt werden. & Ja\\\hline
    T11 & Die auszublendenden Links in der Visualisierung auswählen, diese werden pink gefärbt. Danach in der Toolbar mit \textit{COLLAPSE} ausblenden. & Ja \\\hline
    T12 & In der Navigation im Menu \textit{View}, \textit{New View} auswählen. Die Visualisierung wird aktualisiert mit einer leeren \gls{View}. & Ja\\\hline
    T13 & Eine neue \gls{View} erstellen einen beliebigen \gls{Node} darstellen, danach den Browser schliessen und neu öffnen. Nun muss die zuvor erstellte \gls{View} ebenfalls verfügbar und der eben erstelle \gls{Node} sollte sichtbar sein. & Ja\\\hline
    T14 & Titel eines \gls{Node}[s] in der Datenbasis aktualisieren. Dieser muss auch in der Visualisierung aktualisiert sein. & Ja\\\hline
    T15 & Einen \gls{Node} in der Datenbasis löschen. Dieser muss auch aus allen \gls{View} gelöscht sein. & Ja\\\hline
    T16 & Einen Link aus der Datenbasis löschen. Dieser darf nicht mehr in den \gls{View} dargestellt werden. &Ja\\\hline
    T17 & Einen Link in der Datenbasis erstellen. Dieser darf nicht dargestellt werden, muss aber bei dem entsprechenden \gls{Node} zur Auswahl stehen für die Darstellung. & Ja\\\hline
    T18 & Einen bestehenden \gls{Node} mit mehr als sieben ausgehenden \gls{Node}[s] darstellen. Es dürfen dabei nicht mehr als sieben Links dargestellt werden. & Ja\\\hline
    T19 & In der Toolbar mit \textit{SHOW LINK LABEL} bzw. \textit{HIDE LINK LABEL} die Labels der Links ein oder ausblenden. & Ja\\\hline
    \caption{Testfälle Beschreibung}
  \label{tab:testkonzept-detail}
\end{longtable}



\section{User-Stories}

\begin{longtable}{|p{1cm} | p{10.8cm} |}
\hline
ID  & Description\\ \hline
S1  & Die Schnittstellen und die Model-Klassen für die Implementierung definieren.     \\ \hline
S2  & Das Projektteam erhält einen Überblick über den aktuellen Stand der Technik. Es soll eine Auswahl von vier möglichen Frameworks getroffen werden.\\ \hline
S3  & Detail Beurteilung zweier Frameworks anhand des Standard-Szenarios.\\ \hline
S4  & Detail Beurteilung zweier Frameworks anhand des Standard-Szenarios. \\ \hline
S5  & Definition des Standard-Szenarios zur genaueren Beurteilung der vier möglichen Frameworks. \\ \hline
S6  & Mock-up aller Benutzeroberflächen erstellen \\ \hline
S7  & Anbindung des Interface-Paket (\textit{ikc-visual}) via NPM Setup. Typescript- und React-Setup des Projekts im Gitlab \\ \hline
S8  & Der Benutzer kann einen \gls{Node} z.B. aus der Suche von ausserhalb in die Visualisierung ziehen. Wird der \gls{Node} an einer freien Position losgelassen, wird dieser der Visualisierung an dieser Stelle hinzugefügt. Wird er jedoch über einem bestehenden \gls{Node} losgelassen, wird ein neuer Link zu diesem erstellt und der neue \gls{Node} in seiner Umgebung platziert. Dies soll sowohl im mobilen als auch auf im Desktop-Umfeld möglich sein. Der entsprechende \gls{Node} wird immer zusammen mit all seinen Kindern angezeigt. Sind es mehr als sieben, werden nur sieben angezeigt und der Rest über das Kontext-Menü zugänglich gemacht.\\ \hline
S9  & Einen \gls{Node} in der Visualisierung kann per \gls{Drag'n'Drop} positioniert werden. Diese Position muss gespeichert werden.    \\ \hline
S10 & Werden zwei \gls{Node}[s] in der Visualisierung aufeinander gezogen, sollen zwei beschriftete Links entstehen, jeweils in eine Richtung.       \\ \hline
S11 & Mittels eines Kontext-Menüd kann der Benutzer neue \gls{Node}[s] erstellen und die Visualisierung hinzufügen oder einen bestehenden hinzufügen. Dies geschieht an der Stelle, wo das Menü aufgerufen wurde. Das Menü öffnet sich durch einen Rechtsklick oder einen langen Tap. Das Suchfeld wird vom \gls{ikc-core} geliefert werden. \\ \hline
S12 & Der Benutzer kann durch einen Rechtklick auf einen \gls{Node} oder einen langen Tap das \gls{Node} Kontext-Menü aufrufen. Dort soll er die folgenden Möglichkeiten haben: \textit{Edit Node}, \textit{Hide Node} \textit{LinkToExistingNode}, \textit{LinkToNewNode} (Link zu einem bestehenden \gls{Node} erstellen oder einen neuen \gls{Node} erstellen und dann verlinken, der entsprechende \gls{Node} wird in der Umgebung angezeigt), \textit{CollapseAllLinks} (alle ausgehenden Links verstecken), \textit{ExpandAllLinks} (alle ausgehenden Links anzeigen) \textit{SearchLinks} (Versteckte Links druchsuchen und einzeln anzeigen können, die Suche soll sowohl den Titel des Ziel \gls{Node}[s] als auch das Label des Links verwenden) Das Suchfeld wird vom \gls{ikc-core} geliefert werden. \\ \hline
S13 & Links können separat ausgewählt und ausgeblendet werden.  \\ \hline
S14 & Die Labels der Visualisierung sollen versteckt oder angezeigt werden können.\\ \hline
S15 & Die Visualisierung soll in den bestehenden \gls{ikc-core} integriert werden. \\ \hline
S16 & Änderungen in der Visualisierung sollen in den \gls{ikc-core} als auch umgekehrt vom \gls{ikc-core} in die Visualisierung übernommen werden. \\ \hline
S17 & \gls{View}[s] sollen dauerhaft gespeichert werden. Ebenfalls sollen neue erstellt werden können.          \\ \hline
S18 & Es müssen die folgenden Dialoge bereitgestellt werden, \textit{SearchExistingNodeToConnect}, \textit{NewNodeToConnect} und \textit{NewNode}.\\ \hline
S19 & Es müssen zwei Suchfelder zur Verfügung gestellt werden: (\textit{NodeSearch} und \textit{LinkSearch})\\ \hline
    \caption{User Stories Beschreibung}
\label{user-stories-desc}
\end{longtable}

\section{Wichtigste Protokolle}

\includepdf[pages=-]{210916.pdf}
\includepdf[pages=-]{210916Skype.pdf}
\includepdf[pages=-]{260916.pdf}
\includepdf[pages=-]{41016.pdf}
\includepdf[pages=-]{141016.pdf}

\section{Arbeitsjournal}
\includepdf[pages=-]{Arbeitsjournal.pdf}

\backmatter

%\includepdf[pages={-}]{declaration-originality.pdf}

\end{document}
