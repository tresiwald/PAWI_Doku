\chapter{Einleitung}

Im Forschungsprojekt \textbf{\textit{IKC}}\footnote{\gls{Intuitive Knowledge Connectivity}} wird ein Prototyp für den Umgang mit einem plattformübergreifenden Wissensnetzwerk entwickelt. Obwohl die grundlegende Datenbasis ebenfalls in Form eines Netzwerks aufgebaut ist, wählt die Benutzeroberfläche einen anderen Ansatz: Bisher handelt es sich lediglich um eine technische Konsole. Diese macht für den Benutzer zwar alle existierenden Funktionalitäten zugänglich, jedoch ist sie weder sonderlich effizient noch benutzerfreundlich. Für einen ersten Machbarkeitsnachweis war dies ausreichend. Um den Prototypen aber einem grösseren Nutzerkreis verfügbar zu machen, gibt es Optimierungsbedarf. Auch macht es aus Sicht des Projektteams Sinn, dass die Netzwerkstruktur auch für den Benutzer sichtbar ist. Das Ziel dieser Arbeit ist darum ein Prototyping einer webbasierten Visualisierung zur intuitiven Interaktion mit einem Wissensnetzwerk.

Zum besseren Verständnis werden potentiell unbekannte oder projekt-spezifische Begriffe kursiv und fett dargestellt. Zu jedem Begriff ist eine kurze Beschreibung im Glossar (\autoref{glossar}) am Ende der Dokumentation zu finden. Zusätzlich zu diesem Dokument werden noch die beiden Handbücher für die \textit{Entwicklung} und die \textit{Handhabung} geliefert. 


