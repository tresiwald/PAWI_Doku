\chapter{Einleitung}

Im Hasler-Projekt Intuitive Knowledge Connectivity (IKC) wird ein Prototyp für plattformübergreifendes Wissensnetzwerk erstellt. Die grundlegende Datenbank basiert auf einen Netzwerk (einem gerichteten beschrifteten Property-Graph), hat aber bisher nur eine technische Konsole. Eine ideale grafische Benutzerschnittstelle stellt das Netzwerk oder Teilausschnitte daraus visuell und zweidimensional dar, und stellt beschriftete Knoten und Pfeile grafisch dar. Dies ermöglicht eine einfache und übersichtliche Repräsentation und eine intuitive Interaktion mit den gegebenen Kanten und Knoten. Der bestehende Prototyp implementiert die grundlegenden Datenbankoperationen (C\-R\-U\-D) und die Verknüpfung von Knoten mit Drag and Drop. Auf dieser Kern-Software soll aufgebaut werden, um diese hinsichtlich intuitiver Benutzung zu erweitern, damit mit mehreren Knoten im Netzwerk gleichzeitig und visuell gearbeitet werden kann. Dies soll in erster Linie auf dem Touchscreen (mobil / Tablet) und in zweiter Linie responsive mit dem gleichen Code auch mit Bildschirm, Tastatur und Maus / Touchpad bedienbar sein.
\todo{erweitern, anpassen?}